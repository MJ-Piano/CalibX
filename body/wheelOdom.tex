\section{轮式里程计}
\subsection{编码器原理}
假设固定时间间隔$\mathbf{t}$内左右轮子的转动脉冲数$\mathbf{p}_l, \mathbf{p}_r$。我们记轮子半径为$\mathbf{r}$,每一圈总脉冲为$\mathbf{N}$,因此左右轮子的角速度$\omega_l,\omega_r$分别为,
\begin{equation}
	\left\{
	\begin{aligned}
		& \omega_l=\frac{2 \pi *p_l}{N*t} \\
		& \omega_r=\frac{2 \pi *p_r}{N*t}
	\end{aligned}
	\right.
\end{equation}
注意,由于左右轮由不同的电机控制,因此一般左右轮角速度$\omega_l,\omega_r$除匀速直线运动外一般不相等,是为差速。
根据角速度与线速度的关系$\upsilon=\omega*\mathbf{r}$,左右轮子的线速度$\upsilon_l,\upsilon_r $分别为,
\begin{equation}
	\left\{
	\begin{aligned}
		& v_l=\omega_l*d \\
		& v_r=\omega_r*d
	\end{aligned}
	\right.
\end{equation}
\subsection{运动模型}
\subsubsection{两轮差速底盘模型}

\subsubsection{阿克曼模型}
两轮差分底盘模型或者阿克曼模型,其运动学模型如图\ref{fig:wheel-odom-model}所示,轮子做半径为$\mathbf{R}$的圆周运动,注意,轮子本身也自转。有点类似于地球自转同时也绕太阳公转。
%\begin{figure}[H]
%	\centering
%	\includegraphics[width=0.45\textwidth]{figure/wheel/wheelTop}
%	\includegraphics[width=0.45\textwidth]{figure/wheel/wheelRight}
%	\caption{两轮差分底盘运动学模型}
%	\label{fig:wheel-odom-model}
%\end{figure}
\begin{figure}[H]
	\centering
	\includegraphics[width=0.7\textwidth]{figure/wheel/wheel}
	\caption{两轮差分底盘运动学模型}
	\label{fig:wheel-odom-model}
\end{figure}

\paragraph{解析方法}\mbox{} \\
轨迹的线速度也等于各自轮子轨迹与地面接触处的线速度,即$\upsilon_L = \upsilon_l, \upsilon_R = \upsilon_r$,且左右轮子轨迹的角速度与机体中心的角速度$\omega$相等,即$\omega = \omega_R = \omega_L$,因此,
\begin{equation}
	\begin{aligned}
		\omega = \omega_R = \frac{\upsilon_r}{R+b/2} &= \omega_L = \frac{\upsilon_l}{R-b/2} \\
		\rightarrow v_l(R+b/2) & =v_r(R-b/2) \\
		\rightarrow\left(v_r-v_l\right) R & =\left(v_r+v_l\right) b \\
		\rightarrow R & =\frac{v_r+v_l}{v_r-v_l}*\frac{b}{2} 
	\end{aligned}
\end{equation}
因此机体角速度为,
\begin{equation}
	\begin{aligned}
		\omega &= \frac{\upsilon_r}{R+b/2} \\
		&= \frac{\upsilon_r}{\frac{v_r+v_l}{v_r-v_l}*\frac{b}{2} +b/2} \\
		&= \frac{\upsilon_r}{\frac{v_r}{v_r-v_l} b} \\
		&= \frac{v_r-v_l}{b}
	\end{aligned}
\end{equation}
机体速度为,
\begin{equation}
	\begin{aligned}
		\upsilon &= \omega*r \\
		&= \frac{v_r-v_l}{b} * \frac{v_r+v_l}{v_r-v_l}b \\
		&= \frac{v_r+v_l}{2}
	\end{aligned}
\end{equation}
% 参考:https://xiaotaoguo.com/p/wheel-odometry-model-calibration/
\paragraph{相对速度方法}\mbox{} \\
以左轮为参考系,则右轮相对于左轮的速度为,
\begin{equation}
	v' = v_r-v_l
\end{equation}
由于旋转半径为$b$,则角速度为,
\begin{equation}
	\omega = \frac{v'}{b} = \frac{v_r-v_l}{b} 
\end{equation}
根据左右轮子轨迹的角速度与机体中心的角速度$\omega$相等,即$\omega = \omega_R = \omega_L$,但是机体中心旋转半径仅为右轮的一半,因此,机体中心线速度为
\begin{equation}
	v = \frac{v_r+v_l}{2}
\end{equation}
\paragraph{微积分几何方法}\mbox{} \\
假设在极小时间$t$内,左右轮的运动轨迹如图所示,
\begin{figure}[H]
	\centering
	\includegraphics[width=0.7\textwidth]{figure/wheel/wheelGeometry}
	\caption{两轮差分底盘运动学模型}
	\label{fig:wheel-geometry-model}
\end{figure}
左右轮的近似位移之差$s$为,
\begin{equation}
	s = (v_r-v_l)*t
\end{equation}
根据三角关系,并利用等价无穷小$tan(\theta) \approx \theta $去掉三角符号,
\begin{equation}
	tan(\theta) \approx \theta = \frac{s}{b} = \frac{(v_r-v_l)*t}{b} 
\end{equation}
又因为,
\begin{equation}
	\theta = \omega*t
\end{equation}
因此,也可以的得到
\begin{equation}
	\omega = \frac{v_r-v_l}{b} 
\end{equation}
机体中心的位移为,
\begin{equation}
	s_B = \frac{(v_r+v_l)*t}{2}
\end{equation}
因此机体中心速度为,
\begin{equation}
	v = \frac{s_B}{t} = \frac{(v_r+v_l)}{2}
\end{equation}
%reference: https://en.wikipedia.org/wiki/Differential_wheeled_robot
\subsection{轮式里程计积分}