\section{深度相机}
%A Brief Analysis of the Principles of Depth Cameras: Structured Light, TOF, and Stereo Vision
%https://wiki.dfrobot.com/brief_analysis_of_camera_principles,介绍了各种深度相机的优缺点
%https://hackernoon.com/3-common-types-of-3d-sensors-stereo-structured-light-and-tof-194033f0
\subsection{结构光相机}
Microsoft Kinect V1、Orbbec Astra
\subsubsection{单目结构光相机}
根据论文\cite{xu2022depth},单目结构光相机如Kinect组成如图\ref{fig:mono_structured_light}
\begin{figure}[H]
	\centering
	\includegraphics[width=0.7\textwidth]{figure/strcutured_light/mono_structured_light}
	\caption{单目结构光相机原理}
	\label{fig:mono_structured_light}
\end{figure}
其深度计算公式为,
\begin{equation}
	Z_m=\frac{Z_{r e f}}{1-\frac{Z_{r e f} d_m}{B_m f_m}}
\end{equation}
\subsubsection{被动双目相机}
被动双目相机如Zed
\subsubsection{主动双目结构光相机}
主动双目结构光相机如Intel的D435
\subsection{TOF相机}
飞行时间时测距(ToF:Time of Flight):指通过测量光波飞行的时间来计算距离。ToF传感器通过光源驱动芯片调制信号,调制信号控制激光发射器发射高频调制的近红外光,这些红外光被物体表面漫反射后,接收器通过发射光与接收光的相位差或者时间差来计算深度信息。根据飞行时间差的获取方法不同,分为dToF和iToF两类。

\subsubsection{dToF}
dToF(direct Time-of-Flight)即直接ToF,指直接测量光脉冲的发射和接收飞行时间差,从而计算出实际距离,如图\ref{fig:dtof_camera}所示。
\begin{figure}[H]
	\centering
	\includegraphics[width=0.7\textwidth]{figure/tof/dtof}
	\caption{tof相机原理}
	\label{fig:dtof_camera}
\end{figure}
即距离为,
\begin{equation}
	d = \frac{c\Delta t}{2}
\end{equation}
\subsubsection{iToF}
iToF(indirect Time-of-Flight)即间接ToF,指发射调制光波,通过测量反射光波的相位偏移,间接地计算发射光波和接收光波的时间差,从而计算出实际距离。根据调制方法的不同,又可以分为脉冲调制(Pulsed Modulation)和连续波调制(ContinuousWave Modulation)两种方式。
\paragraph{脉冲调制iToF}\mbox{} \\
\begin{figure}[H]
	\centering
	\includegraphics[width=0.7\textwidth]{figure/tof/pl_itof}
	\caption{脉冲调制光波itof原理计算相位差}
	\label{fig:pl-itof}
\end{figure}
\begin{figure}[H]
	\centering
	\includegraphics[width=0.7\textwidth]{figure/tof/pl_itof_logic}
	\caption{脉冲调制光波itof原理计算相位差}
	\label{fig:pl-itof-logic}
\end{figure}
%https://blog.csdn.net/flow_specter/article/details/122730446
发射光波的脉冲宽度为$T_0$,相位时延为$T_d$,则收到的光子数满足
\begin{equation}
	\begin{aligned}
		C_0 - C_b &= K(T_0 - T_d) \\
		C_1 - C_b & = KT_d
	\end{aligned}
\end{equation}
消去$K$有,
\begin{equation}
	\begin{aligned}
		& C_0 - C_b = \frac{C_1 - C_b}{T_d}(T_0 - T_d) \\
		& \Rightarrow T_d(C_0 - C_b) = (C_1 - C_b)(T_0 - T_d) \\
		& \Rightarrow T_d\left((C_0 - C_b) + (C_1 - C_b)\right) = (C_1 - C_b)T_0 \\
		& \Rightarrow T_d = \frac{(C_1 - C_b)T_0}{C_0 + C_1 - 2C_b}
	\end{aligned}
\end{equation}
则距离为,
\begin{equation}
	\begin{aligned}
		d = \frac{cT_d}{2} = \frac{cT_0(C_1 - C_b)}{C_0 + C_1 - 2C_b}
	\end{aligned}
\end{equation}
\paragraph{连续波调制iToF}\mbox{} \\
如图\ref{fig:itof_camera}所示。
%Microsoft Azure Kinect、 ASUS Xtion 2
%参考<Time-of-Flight and Kinect Imaging>
\begin{figure}[H]
	\centering
	\includegraphics[width=0.7\textwidth]{figure/tof/tof}
	\caption{tof相机原理}
	\label{fig:itof_camera}
\end{figure}
发射端正弦信号为,
\begin{equation}
	g(t)=\cos (\omega t)
\end{equation}
信号被物体反射,并被接收机接到,
\begin{equation}
	s(t)=b+a \cos (\omega t+\phi)
\end{equation}
接收信号与发射信号做互相关,
\begin{equation}
	\begin{aligned}
		c(\tau) &=s * g=\int_{-\infty}^{\infty} s(t) \cdot g(t+\tau) d t \\
		&= \int_{-\infty}^{\infty} \left(b+a \cos (\omega t+\phi)\right) \cdot \cos\left(\omega(t+\tau)\right) d t \\
		&\xlongequal{\text{公式}(\ref{equ:cross-product-cp})} \int_{0}^{T} \left(b+a \cos (\omega t+\phi)\right) \cdot \cos\left(\omega(t+\tau)\right) d t \\
		&\xlongequal{\text{公式}(\ref{equ:product2sum})} \int_{0}^{T} \left(b\cos\left(\omega(t+\tau)\right) + \frac{a}{2}\left[\cos (2\omega t + \phi + \omega\tau) + \cos\left(\phi - \omega\tau\right) \right]  \right)d t \\
		&\xlongequal{\text{公式}(\ref{equ:integrate_cos})}\frac{b}{\omega}\sin \left(\omega(t+\tau)\right)\big|_{0}^{T} + \frac{a}{4\omega}\sin (2\omega t + \phi + \omega\tau)\big|_{0}^{T}\\
		&\xlongequal{T=\frac{1}{\omega}} \frac{b}{\omega}\left(\sin \left(1 + \omega\tau \right)- \sin \left(\omega\tau\right) \right) + \frac{a}{4\omega}\left(\sin (2 + \phi + \omega\tau) - \sin (\phi + \omega\tau)  \right)\\
		&\xlongequal{\text{公式}(\ref{equ:sum2product})}\frac{b}{\omega}2\cos\frac{2\omega\tau+1}{2} \sin \frac{1}{2} + \frac{a}{2\omega}\left( \cos(1 + \phi + \omega\tau) \sin(1)\right)\\%todo:false?
		&= \frac{a}{2} \cos (\omega \tau+\phi)+b
	\end{aligned}
\end{equation}
其中,
\begin{itemize}
	\item $b$是固定偏置
	\item $a$是幅度
	\item $\phi$是相移
	\item $\tau$是内部偏移
\end{itemize}
%%https://zhuanlan.zhihu.com/p/648858398
%https://blog.csdn.net/flow_specter/article/details/122730446
对$c(\tau)$进行四次连续等间隔为$\frac{\pi}{2}$采样,
\begin{figure}[H]
	\centering
	\includegraphics[width=0.7\textwidth]{figure/tof/cw_itof}
	\caption{连续光波itof原理计算相位差}
	\label{fig:cw-itof}
\end{figure}
即,
\begin{equation}
	A_i=c(i \cdot \pi / 2), \quad i=0, \ldots, 3
\end{equation}
即,
\begin{equation}
	\begin{aligned}
		& A_0=a\sin(0+\phi) + b = a\sin(\phi) + b\\
		& A_1=a\sin(\frac{\pi}{2}+\phi) + b = a\cos(\phi) + b\\
		& A_2=a\sin(\pi+\phi) + b = -a\sin(\phi) + b\\
		& A_3=a\sin(\frac{3\pi}{2}+\phi) + b = -a\cos(\phi) + b\\
	\end{aligned}
\end{equation}
即相移和振幅,
\begin{equation}
	\begin{aligned}
		& \phi = \arctan\left(\frac{2a\sin(\phi)}{2a\cos(\phi)}\right) =\arctan\left(\frac{A_0-A_2}{ A_1-A_3}\right) \\
		& a= \frac{\sqrt{\left(2a\sin(\phi)\right)^2 + \left(2a\cos(\phi)\right)^2}}{2} =  \frac{\sqrt{\left(A_0-A_2\right)^2+\left(A_1-A_3\right)^2}}{2}
	\end{aligned}
\end{equation}
则距离为,
\begin{equation}
	d=\frac{c}{4 \pi \omega} \phi
\end{equation}

