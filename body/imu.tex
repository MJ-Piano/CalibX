\section{IMU}
微机电(Micro-electromechanical Systems, MEMS)IMU以其便宜、便携、低功耗等优点被广泛应用于汽车、消费电子、机器人等行业。本章仅介绍MESM IMU。
\subsection{IMU}
IMU由加速度计和陀螺仪组成,有的还包含磁力计。根据论文\cite{woodman2007introduction},陀螺仪的误差一般包含以下部分,
\begin{itemize}
	\item \textbf{零偏}(Constant Bias)。陀螺仪的零偏是陀螺仪不进行任何旋转时的平均输出,是一个固定的偏置
	\item \textbf{热机械白噪声}(Thermo-Mechanical White Noise)。也叫角度随机游走(Angle Random Walk, ARW)
	\item \textbf{零偏不稳定性}(Bias Instability)。也叫闪烁噪声(Flicker Noise)
	\item \textbf{温度效应}
	\item \textbf{标定误差}
\end{itemize}
类似地,加速度计的误差一般包含以下部分,
\begin{itemize}
	\item \textbf{零偏}(Constant Bias)。陀螺仪的零偏是陀螺仪不进行任何旋转时的平均输出,是一个固定的偏置
	\item \textbf{热机械白噪声}(Thermo-Mechanical White Noise)。也叫速度随机游走(Velocity Random Walk, VRW)
	\item \textbf{零偏不稳定性}(Bias Instability)。也叫闪烁噪声(Flicker Noise)
	\item \textbf{温度效应}
	\item \textbf{标定误差}
\end{itemize}
\subsubsection{IMU简单建模}
\begin{figure}[H]
	\centering
	\includegraphics[width=0.5\textwidth]{figure/imu/imu_simple}
	\caption{IMU简单建模}
	\label{fig:imu-simple}
\end{figure}
IMU的测量模型简单建模如图\ref{fig:imu-simple}所示,IMU的加速度和陀螺仪完美重合,其坐标系与本体坐标系B重合。则加速度测量模型为,
\begin{equation}\label{equ:acc-mea-model-simple}
^B\mathbf{a}_m={_W^B}\mathbf{R}\left({^W}\mathbf{a}-{^W}\mathbf{g}\right)+{^B}\mathbf{a}_b+{^B}\mathbf{a}_n
\end{equation}
角速度测量模型为,
\begin{equation}\label{equ:gyr-mea-model-simple}
^B\boldsymbol{\omega}_m ={^B}\boldsymbol{\omega}+{^B}\boldsymbol{\omega}_b+{^B}\boldsymbol{\omega}_n
\end{equation}
其中,${_W^B}\mathbf{R}$为惯性坐标系$W$到IMU本体坐标系$B$的旋转,${^W}\mathbf{g}$为惯性系下的重力加速度,${^W}\mathbf{a}$为惯性系下加速度的真值,${^B}\mathbf{\omega}$为本体坐标系下角速度的真值,${^B}\mathbf{a}_b$为加速度的零漂bias,${^B}\mathbf{\omega}_b$为角速度的零漂bias。零漂bias为随机游走,其一阶导服从高斯分布,即(省略坐标系符号),
\begin{equation}\label{equ:imu-bias-model}
	\begin{aligned}
		\dot{\mathbf{a}}_b &\sim N\left(0, \sigma_{\mathbf{a}_b}^2\right) \\
		\dot{\boldsymbol{\omega}}_b &\sim N\left(0, \sigma_{\boldsymbol{\omega}_b}^2\right) 
	\end{aligned}
\end{equation}
${^B}\mathbf{a}_n, {^B}\mathbf{\omega}_n$分别为加速度、角速度的高斯白噪声,即(省略坐标系符号),
\begin{equation}\label{equ:imu-gaussian-noise}
	\begin{aligned}
		\mathbf{a}_n &\sim N\left(0, \sigma_{\mathbf{a}_n}^2\right) \\
		\boldsymbol{\omega}_n &\sim N\left(0, \sigma_{\boldsymbol{\omega}_n}^2\right)
	\end{aligned}
\end{equation}
其中,$\sigma_{\mathbf{a}_b}, \sigma_{\boldsymbol{\omega}_b}, \sigma_{\mathbf{a}_n}, \sigma_{\boldsymbol{\omega}_n}$一般称作IMU的噪声参数。
%%%%%%%%%%%%%%%%%%%%%%%%%%%%%%%%%%%%%%%%
\subsubsection{IMU复杂建模}
\noindent\textbf{calibrated模型}\\
根据\cite{rehder2016extending},一般情况下(如多IMU),IMU测量坐标系A(即IMU测量数据的坐标系)与本体坐标系B不一定重合,如图\ref{fig:acc_calibrated}所示。
\begin{figure}[H]
	\centering
	\includegraphics[width=0.8\textwidth]{figure/imu/acc_calibrated}
	\caption{IMU加速度计calibrated模型}
	\label{fig:acc_calibrated}
\end{figure}
以加速度计为例,其测量模型为,
\begin{equation}\label{equ:acc_calibrated}
	^A\mathbf{a}_m={_B^A}\mathbf{R} \cdot {_W^B}\mathbf{R}\left({^W}\mathbf{a}-{^W}\mathbf{g}\right) + \left\lfloor{ }^B \dot{\boldsymbol{\omega}}\right\rfloor_{\times} {^B}\mathbf{r}_{A} + \left\lfloor{ }^B \boldsymbol{\omega}\right\rfloor_{\times}^2{ }^B \mathbf{r}_{A} +{^A}\mathbf{a}_b+{^A}\mathbf{a}_n
\end{equation}
其中,${_B^A}\mathbf{R}$是本体坐标系B到IMU测量坐标系A的旋转,${ }^B \mathbf{r}_{A}$是IMU测量坐标系A在本体坐标系B下的平移,${ }^B\boldsymbol{\omega}$是本体坐标系下的角速度。该模型称为calibrated模型\cite{rehder2016extending}。类似地,假设加速度计与陀螺仪完美重合,则陀螺仪的测量模型为,
\begin{equation}\label{equ:gyr_calibrated}
	^A\boldsymbol{\omega}_m = {_B^A}\mathbf{R} \cdot {^B}\boldsymbol{\omega}+{^A}\boldsymbol{\omega}_b+{^A}\boldsymbol{\omega}_n
\end{equation}
%其中,${_B^A}\mathbf{R}_{\omega}$是本体坐标系B到IMU陀螺仪测量坐标系A的旋转。\\
\noindent\textbf{scale-misalignment模型}\\
calibrated模型过于简单,对于低成本的消费级IMU来说,还存在轴偏和尺度误差。轴偏如图\ref{fig:acc_scaled_mis}所示。
\begin{figure}[H]
	\centering
	\includegraphics[width=0.8\textwidth]{figure/imu/acc_scaled_mis}
	\caption{IMU加速度计scale-misalignment模型}
	\label{fig:acc_scaled_mis}
\end{figure}
其测量模型为,
\begin{equation}\label{equ:acc_scaled_mis}
	\begin{aligned}
		^A\mathbf{a}_m&=\mathbf{S}_{\alpha} \cdot \mathbf{M}_{\alpha} \cdot {_B^A}\mathbf{R} \cdot \left( {_W^B}\mathbf{R}\left({^W}\mathbf{a}-{^W}\mathbf{g}\right) + \left\lfloor{ }^B \dot{\boldsymbol{\omega}}\right\rfloor_{\times} {^B}\mathbf{r}_{A} + \left\lfloor{ }^B \boldsymbol{\omega}\right\rfloor_{\times}^2{ }^B \mathbf{r}_{A} \right) +{^A}\mathbf{a}_b+{^A}\mathbf{a}_n \\
		&\triangleq \mathbf{S}_{\alpha} \cdot \mathbf{M}_{\alpha} \cdot {^A}\mathbf{a}_{\alpha} + {^A}\mathbf{a}_b+{^A}\mathbf{a}_n
	\end{aligned}
\end{equation}
其中,$\mathbf{S}_{\alpha}$是对角矩阵,对角元素分别表示各个轴的尺度因子。$\mathbf{M}_{\alpha}$是下三角矩阵,对角线为1、对角线下元素有非零值,表示小的角度误差,即,
\begin{equation}\label{equ:misalign_matrix}
	\mathbf{M}_{\alpha}=
	\left[\begin{array}{ccc}
		1 & 0 & 0 \\
		\alpha_{x z} & 1 & 0 \\
		\alpha_{x y} & \alpha_{y x} & 1
	\end{array}\right]
\end{equation}
轴偏的示意如图\ref{fig:axis_misalign}所示。%注意:论文rehder2016extending关于M的描述不准确,M的对角线应该是1
\begin{figure}[H]
	\centering
	\includegraphics[width=0.8\textwidth]{figure/imu/axis_misalign}
	%注意:这个图修正了论文\ref{fig:axis_misalign}的一个角度标志的错误
	\caption{轴偏模型。正交坐标系A(红色)与测量坐标系S原点重合。正交坐标系A的各个轴分别绕另外两个轴旋转(蓝色虚线)小角度后得到测量坐标系S。其中,转动的方向由箭头起终点表示,转动的角度为位于箭头起点的$\beta$值}
	\label{fig:axis_misalign}
\end{figure}
根据\cite{tedaldi2014robust},在小角度误差的前提下,非正交的IMU测量坐标系S的测量$\mathbf{s}^S$能够通过以下转换到正交坐标系A,
\begin{equation}
	\mathbf{s}^B={_S^B}\mathbf{T} \mathbf{s}^S, \quad {_S^B}\mathbf{T}=\left[\begin{array}{ccc}
		1 & -\beta_{y z} & \beta_{z y} \\
		\beta_{x z} & 1 & -\beta_{z x} \\
		-\beta_{x y} & \beta_{y x} & 1
	\end{array}\right]
\end{equation}
%参考:Quaternion kinematics for the error-state Kalman filter关于局部扰动
其中,$\beta_{i j}$表示测量坐标系S的第$i$个轴相对于正交坐标系A的旋转角度。若正交坐标系A与测量坐标系S满足以下条件:
%参考:https://zhuanlan.zhihu.com/p/639114848
\begin{itemize}
	\item [(1)]正交坐标系A的z轴与测量坐标系S的z轴重合
	\item [(2)]正交坐标系A的y轴位于测量坐标系S的y、z轴组成的平面内
\end{itemize}
则根据条件(1),$\beta_{z y} = \beta_{z x} = 0$;根据条件(2)则$\beta_{y z} = 0$,即可得到如图\ref{fig:axis_misalign_z}所示的简化轴偏模型。
\begin{figure}[H]
	\centering
	\includegraphics[width=0.8\textwidth]{figure/imu/axis_misalign_z}
	%注意:这个图修正了论文\ref{fig:axis_misalign}的一个角度标志的错误
	\caption{轴偏简化模型。正交坐标系A的z轴与测量坐标系S的z轴重合且正交坐标系A的y轴位于测量坐标系S的y、z轴组成的平面内}
	\label{fig:axis_misalign_z}
\end{figure}
因此,非正交的IMU测量坐标系S的到正交坐标系A的变换${_S^B}\mathbf{T}$简化为,
\begin{equation}
	 {_S^B}\mathbf{T}=
	 \left[\begin{array}{ccc}
		1 & 0 & 0 \\
		\beta_{x z} & 1 & 0 \\
		-\beta_{x y} & \beta_{y x} & 1
	\end{array}\right]
\end{equation}
忽略正负号或者改变$\beta_{x y}$的旋转方向即可得到公式(\ref{equ:misalign_matrix})中的$\mathbf{M}_{\alpha}$。类似地,陀螺仪的测量模型为,
\begin{equation}\label{equ:gyr_scaled_mis}
	^A\boldsymbol{\omega}_m = \mathbf{S}_{\omega} \cdot \mathbf{M}_{\omega} \cdot {_B^A}\mathbf{R} \cdot  {^B}\boldsymbol{\omega}+\mathbf{A}_{\omega} {^A}\mathbf{a}_{\alpha} +{^A}\boldsymbol{\omega}_b+{^A}\boldsymbol{\omega}_n
\end{equation}
%注意:这里{_B^A}\mathbf{R}与论文rehder2016extending的描述不符,即仅有一个{_B^A}\mathbf{R},这与代码实现一致
其中,$\mathbf{S}_{\omega}, \mathbf{M}_{\omega}$分别表示陀螺仪尺度因子和轴偏的矩阵,其定义与形式与加速度计一致。$\mathbf{A}_{\omega}$用来建模加速度导致的角速度偏移(g-sensitivity),一般是满填充矩阵。
%%%%%%%%%%%%%%%%%%%%%%%%%%%%%%%%%%%%%%%%
\noindent\textbf{scale-misalignment-size-effect模型}\\
对于高端的IMU,加速度计和陀螺仪的三轴是分立的,如图\ref{fig:acc_size_effect}所示。
\begin{figure}[H]
	\centering
	\includegraphics[width=0.8\textwidth]{figure/imu/acc_size_effect}
	\caption{IMU加速度计scale-misalignment-size-effect模型}
	\label{fig:acc_size_effect}
\end{figure}
令$\mathbf{I}_x=\left[1, 0, 0\right]^T, \mathbf{I}_y=\left[0, 1, 0\right]^T, \mathbf{I}_z=\left[0, 0, 1\right]^T$,其测量模型为,
\begin{equation}\label{equ:acc_size_effect}
%	\begin{aligned}
	\begin{split}
		^A\mathbf{a}_m
		&=\mathbf{S}_{\alpha} \cdot \mathbf{M}_{\alpha} \cdot {_B^A}\mathbf{R}_{\alpha} \cdot \left[ {_W^B}\mathbf{R}\left({^W}\mathbf{a}-{^W}\mathbf{g}\right) + \mathbf{I}_x\left( \left\lfloor{ }^B \dot{\boldsymbol{\omega}}\right\rfloor_{\times} {^B}\mathbf{r}_{x} + \left\lfloor{ }^B \boldsymbol{\omega}\right\rfloor_{\times}^2{ }^B \mathbf{r}_{x} \right) \right.\\ 
		&\left. +\mathbf{I}_y\left( \left\lfloor{ }^B \dot{\boldsymbol{\omega}}\right\rfloor_{\times} {^B}\mathbf{r}_{y} + \left\lfloor{ }^B \boldsymbol{\omega}\right\rfloor_{\times}^2{ }^B \mathbf{r}_{y} \right) + \mathbf{I}_z\left( \left\lfloor{ }^B \dot{\boldsymbol{\omega}}\right\rfloor_{\times} {^B}\mathbf{r}_{z} + \left\lfloor{ }^B \boldsymbol{\omega}\right\rfloor_{\times}^2{ }^B \mathbf{r}_{z} \right) \right] +{^A}\mathbf{a}_b+{^A}\mathbf{a}_n \\
		%%
		&= \mathbf{S}_{\alpha} \cdot \mathbf{M}_{\alpha} \cdot {_B^A}\mathbf{R}_{\alpha} \cdot \left[ {_W^B}\mathbf{R}\left({^W}\mathbf{a}-{^W}\mathbf{g}\right) + diag\left( \left\lfloor{ }^B \dot{\boldsymbol{\omega}}\right\rfloor_{\times} {^B}\mathbf{R}_{xyz} + \left\lfloor{ }^B \boldsymbol{\omega}\right\rfloor_{\times}^2 {^B}\mathbf{R}_{xyz} \right) \right] +{^A}\mathbf{a}_b+{^A}\mathbf{a}_n
	\end{split} 
%	\end{aligned}
\end{equation}
其中,$^B\mathbf{R}_{xyz} = \left[{^B}\mathbf{r}_{x}, {^B}\mathbf{r}_{y}, {^B}\mathbf{r}_{z} \right]$分别表示加速度计xyz轴在本体坐标系下的平移,$diag()$表示提取矩阵对角元素的运算符。各轴分立的陀螺仪本文不考虑。%todo
%%%%%%%%%%%%%%%%%%%%%%%%%%%%%%%%%%%%%%%%
\subsection{静态噪声标定}
IMU噪声标定即标定IMU的高斯噪声(\ref{equ:imu-gaussian-noise})和一阶零漂(
\ref{equ:imu-bias-model})。一般采用Alan方差法\cite{woodman2007introduction}标定。以陀螺仪的$x$轴数据为例,计算Allan方差图的算法为\ref{alg:allan-deviation},\\
\begin{algorithm}[H]
	\caption{Allan方差分析法噪声标定}%算法名字
	\label{alg:allan-deviation}
	\LinesNumbered %要求显示行号
	\KwIn{长时间静态IMU数据的角速度$x$轴分量$\{\boldsymbol{\omega}_x\}$}%输入参数
	\KwOut{噪声参数$\sigma_{\boldsymbol{\omega}_{nx}}, \sigma_{\boldsymbol{\omega}_{bx}}$}%输出
	设置最大分段数$N$,计算最小分段长度$\Delta t = \frac{t_{max}-t_{min}}{N}$\;
	\For{$i=1; i \le N; i++$}{
		设置每分段数据长度$t=i*\Delta t$以及分段数$n=\frac{t_{max}-t_{min}}{t}$\;
		根据测量的时间戳,以$t$为长度,将$\{\boldsymbol{\omega}_x\}$数据分成$n$段\;
		计算每个分段内测量数据的平均值$\{\bar{\boldsymbol{\omega}}_{x,1}, \bar{\boldsymbol{\omega}}_{x,2},\cdots,\bar{\boldsymbol{\omega}}_{x,n}\}$\;
		计算Allan方差$AVAR(t)=\frac{1}{2(n-1)}\sum_{j=1}^{n-1}\left(\boldsymbol{\omega}_{x,j+1}-\boldsymbol{\omega}_{x,j}\right)^2$\;
		计算标准差$AD(t)=\sqrt{AVAR(t)}$\;
	}
	以$log(t)$为横轴,$log(AD(t))$为纵轴画Allan方差图
\end{algorithm}
一个可能得到Allan方差图如图\ref{fig:imu-allan}所示。
\begin{figure}[H]
	\centering
	\includegraphics[width=0.8\textwidth]{figure/imu/allan}
	\caption{对数坐标IMU Allan方差分析图}
	\label{fig:imu-allan}
\end{figure}
\subsection{独立IMU内参标定}
%https://tianchi.aliyun.com/forum/post/77528

\subsection{IMU预积分}
%参考《预积分总结与公式推导》
\begin{equation}
	\begin{aligned}
		\mathbf{R}(t+\Delta t) & =\mathbf{R}(t) \cdot \operatorname{Exp}(\boldsymbol{\omega}(t) \cdot \Delta t) \\
		& =\mathbf{R}(t) \cdot \operatorname{Exp}\left(\left(\tilde{\boldsymbol{\omega}}(t)-\mathbf{b}_g(t)-\boldsymbol{\eta}_{g d}(t)\right) \cdot \Delta t\right) \\
		\mathbf{v}(t+\Delta t) & =\mathbf{v}(t)+\mathbf{a}^w(t) \cdot \Delta t \\
		& =\mathbf{v}(t)+\mathbf{R}(t) \cdot\left(\tilde{\mathbf{f}}(t)-\mathbf{b}_a(t)-\boldsymbol{\eta}_{a d}(t)\right) \cdot \Delta t+\mathbf{g} \cdot \Delta t \\
		\mathbf{p}(t+\Delta t) & =\mathbf{p}(t)+\mathbf{v}(t) \cdot \Delta t+\frac{1}{2} \mathbf{a}^w(t) \cdot \Delta t^2 \\
		& =\mathbf{p}(t)+\mathbf{v}(t) \cdot \Delta t+\frac{1}{2}\left[\mathbf{R}(t) \cdot\left(\tilde{\mathbf{f}}(t)-\mathbf{b}_a(t)-\boldsymbol{\eta}_{a d}(t)\right)+\mathbf{g}\right] \cdot \Delta t^2 \\
		& =\mathbf{p}(t)+\mathbf{v}(t) \cdot \Delta t+\frac{1}{2} \mathbf{g} \cdot \Delta t^2+\frac{1}{2} \mathbf{R}(t) \cdot\left(\tilde{\mathbf{f}}(t)-\mathbf{b}_a(t)-\boldsymbol{\eta}_{a d}(t)\right) \cdot \Delta t^2
	\end{aligned}
\end{equation}
假设$\Delta t$恒定,上述方程可以简化为,
\begin{equation}
	\begin{aligned}
		& \mathbf{R}_{k+1}=\mathbf{R}_k \cdot \operatorname{Exp}\left(\left(\tilde{\boldsymbol{\omega}}_k-\mathbf{b}_k^g-\boldsymbol{\eta}_k^{g d}\right) \cdot \Delta t\right) \\
		& \mathbf{v}_{k+1}=\mathbf{v}_k+\mathbf{R}_k \cdot\left(\tilde{\mathbf{f}}_k-\mathbf{b}_k^a-\boldsymbol{\eta}_k^{a d}\right) \cdot \Delta t+\mathbf{g} \cdot \Delta t \\
		& \mathbf{p}_{k+1}=\mathbf{p}_k+\mathbf{v}_k \cdot \Delta t+\frac{1}{2} \mathbf{g} \cdot \Delta t^2+\frac{1}{2} \mathbf{R}_k \cdot\left(\tilde{\mathbf{f}}_k-\mathbf{b}_k^a-\boldsymbol{\eta}_k^{a d}\right) \cdot \Delta t^2
	\end{aligned}
\end{equation}
利用欧拉积分,
\begin{equation}
	\begin{aligned}
		\mathbf{R}_j & =\mathbf{R}_i \cdot \prod_{k=i}^{j-1} \operatorname{Exp}\left(\left(\tilde{\mathbf{\omega}}_k-\mathbf{b}_k^g-\boldsymbol{\eta}_k^{g d}\right) \cdot \Delta t\right) \\
		\mathbf{v}_j & =\mathbf{v}_i+\mathbf{g} \cdot \Delta t_{i j}+\sum_{k=i}^{j-1} \mathbf{R}_k \cdot\left(\tilde{\mathbf{f}}_k-\mathbf{b}_k^a-\boldsymbol{\eta}_k^{a d}\right) \cdot \Delta t \\
		\mathbf{p}_j & =\mathbf{p}_i+\sum_{k=i}^{j-1} \mathbf{v}_k \cdot \Delta t+\frac{j-i}{2} \mathbf{g} \cdot \Delta t^2+\frac{1}{2} \sum_{k=i}^{j-1} \mathbf{R}_k \cdot\left(\tilde{\mathbf{f}}_k-\mathbf{b}_k^a-\boldsymbol{\eta}_k^{a d}\right) \cdot \Delta t^2 \\
		& =\mathbf{p}_i+\sum_{k=i}^{j-1}\left[\mathbf{v}_k \cdot \Delta t+\frac{1}{2} \mathbf{g} \cdot \Delta t^2+\frac{1}{2} \mathbf{R}_k \cdot\left(\tilde{\mathbf{f}}_k-\mathbf{b}_k^a-\boldsymbol{\eta}_k^{a d}\right) \cdot \Delta t^2\right] \\
		\text { 其中 } \Delta t_{i j} & =\sum_{k=i}^{j-1} \Delta t=(j-i) \Delta t
	\end{aligned}
\end{equation}
其中,$\Delta t_{i j}=\sum_{k=i}^{j-1} \Delta t=(j-i) \Delta t$。为了避免每次更细$\mathbf{R}_i,\mathbf{v}_i, \mathbf{p}_i$都需要重新积分求解$\mathbf{R}_j,\mathbf{v}_j, \mathbf{p}_j$,引入预积分项,
\begin{equation}
	\begin{aligned}
		\Delta \mathbf{R}_{i j} & \triangleq \mathbf{R}_i^T \mathbf{R}_j \\
		& =\prod_{k=i}^{j-1} \operatorname{Exp}\left(\left(\tilde{\mathbf{o}}_k-\mathbf{b}_k^g-\boldsymbol{\eta}_k^{g d}\right) \cdot \Delta t\right) \\
		\Delta \mathbf{v}_{i j} & \triangleq \mathbf{R}_i^T\left(\mathbf{v}_j-\mathbf{v}_i-\mathbf{g} \cdot \Delta t_{i j}\right) \\
		& =\sum_{k=i}^{j-1} \Delta \mathbf{R}_{i k} \cdot\left(\tilde{\mathbf{f}}_k-\mathbf{b}_k^a-\boldsymbol{\eta}_k^{a d}\right) \cdot \Delta t \\
		\Delta \mathbf{p}_{i j} & \triangleq \mathbf{R}_i^T\left(\mathbf{p}_j-\mathbf{p}_i-\mathbf{v}_i \cdot \Delta t_{i j}-\frac{1}{2} \mathbf{g} \cdot \Delta t_{i j}^2\right) \\
		& =\sum_{k=i}^{j-1}\left[\Delta \mathbf{v}_{i k} \cdot \Delta t+\frac{1}{2} \Delta \mathbf{R}_{i k} \cdot\left(\tilde{\mathbf{f}}_k-\mathbf{b}_k^a-\boldsymbol{\eta}_k^{a d}\right) \cdot \Delta t^2\right]
	\end{aligned}
\end{equation}
其中,$\Delta \mathbf{p}_{i j}$的推导如下,
\begin{equation}
	\begin{aligned}
		& \mathbf{p}_j-\mathbf{p}_i-\mathbf{v}_i \cdot \Delta t_{i j}-\frac{1}{2} \mathbf{g} \cdot \Delta t_{i j}^2 \\
		= & \sum_{k=i}^{j-1}\left(\mathbf{v}_k \cdot \Delta t+\frac{1}{2} \mathbf{g} \cdot \Delta t^2+\frac{1}{2} \mathbf{R}_k \xi_k \cdot \Delta t^2\right)-\sum_{k=i}^{j-1} \mathbf{v}_i \cdot \Delta t-\frac{(j-i)^2}{2} \mathbf{g} \cdot \Delta t^2 \\
		= & \sum_{k=i}^{j-1}\left(\mathbf{v}_k-\mathbf{v}_i\right) \Delta t+\left[\frac{j-i}{2}-\frac{(j-i)^2}{2}\right] \mathbf{g} \cdot \Delta t^2+\frac{1}{2} \sum_{k=i}^{j-1} \mathbf{R}_k \xi_k \cdot \Delta t^2 \\
		= & \sum_{k=i}^{j-1}\left(\mathbf{v}_k-\mathbf{v}_i\right) \Delta t-\sum_{k=i}^{j-1}(k-i) \mathbf{g} \cdot \Delta t^2+\frac{1}{2} \sum_{k=i}^{j-1} \mathbf{R}_k \xi_k \cdot \Delta t^2 \\
		= & \sum_{k=i}^{j-1}\left\{\left[\mathbf{v}_k-\mathbf{v}_i-(k-i) \mathbf{g} \cdot \Delta t\right] \Delta t+\frac{1}{2} \mathbf{R}_k \boldsymbol{\xi}_k \cdot \Delta t^2\right\} \\
		= & \sum_{k=i}^{j-1}\left[\left(\mathbf{v}_k-\mathbf{v}_i-\mathbf{g} \cdot \Delta t_{i k}\right) \Delta t+\frac{1}{2} \mathbf{R}_k \boldsymbol{\xi}_k \cdot \Delta t^2\right] \\
		= & \sum_{k=i}^{j-1}\left[\mathbf{R}_i \cdot \Delta \mathbf{v}_{i k} \cdot \Delta t+\frac{1}{2} \mathbf{R}_k \boldsymbol{\xi}_k \cdot \Delta t^2\right]
	\end{aligned}
\end{equation}

\subsection{读懂IMU数据手册}
%http://i2nav.cn/index/newListDetail_zw.do?newskind_id=13a8654e060c40c69e5f3d4c13069078&newsinfo_id=d4fc52fc6eff4be5859e18406b62d46d
以BMI088为例,其加速度计的技术参数为,
\begin{figure}[h]
	\centering
	\includegraphics[width=0.9\textwidth]{figure/imu/bmi088-acc-spec}
	\caption{BMI088加速度计技术参数}
	\label{fig:bmi088-acc-spec}
\end{figure}
其中,
\begin{itemize}
	\item 量程(Range)指可测量的加速度的最大值,这里的单位是$g$
	\item 灵敏度(Sensitivity)指标度因子,ADC的输出量乘以灵敏度即为测量值。灵敏度与量程的关系为:$\text{灵敏度} = \frac{2^\text{ADC位数}}{\text{满量程}}$。这里,BMI088加速度计的ADC是16位,当量程档设置为$\pm 3g$时,满量程是$6g$,则灵敏度为$\frac{2^{16}}{6}=10922$(理论值与表格的典型值可能有微小差别)。这里的单位LSB(Least Significant Bit)指最低有效位
	\item 灵敏度温漂(Sensitivity Temperature Drift)
	\item 零重力偏置(Zero-g Offset)指加速度计的零偏(也叫bias, bias repeatability等),即加速度计在零重力环境下的输出,理想情况下应该为0
	\item 零重力偏置温漂(Zero-g Offset Temperature Drift)指加速度计零偏的温度漂移,即零偏随稳定变化的变化量
	\item 输出帧率(Out Data Rate)指加速度计测量输出帧率。对于BMI088,最低为12.5Hz,最高为16000HZ。
	\item 带宽(Bandwidth range)指由输出帧率(ODR)和过采样(over sampling ratio, OSR)共同决定的输出数据低通滤波器对应的3dB带宽截止频率
	\item 非线性(Nonlinearity)指满量程直线拟合后的残差
	\item 输出噪声密度(Output Noise Density)指加速度计输出的白噪声,也叫速度随机游走(Velocity Random Walk, VRW)
	\item 跨轴灵敏度(Cross Axis Sensitivity)指任意两轴之间相互影响程度
	\item 对齐误差(Alignment Error)指相对于外封装的对齐误差
\end{itemize}
陀螺仪的技术参数为,
\begin{figure}[h]
	\centering
	\includegraphics[width=0.9\textwidth]{figure/imu/bmi088-gyr-spec}
	\caption{BMI088陀螺仪技术参数}
	\label{fig:bmi088-gyr-spec}
\end{figure}
与加速度计类似,特别地,陀螺仪还有以下参数需要注意,
\begin{itemize}
%	\item 灵敏度公差(Sensitivity tolerance)
%	\item 灵敏度供电电压漂移()
	\item 零角速度偏置(Zero-rate Offset)
	\item g灵敏度(g-sensitivity)指加速度激励对陀螺仪三轴灵敏度的影响
%	\item 输出帧率
%	\item 带宽
%	\item 非线性,
	\item 输出噪声(Output Noise)指噪声幅度RMS值,与噪声的功率谱密度、者角度随机游走(Angular Random Walk, ARW)或者噪声密度(Rate Noise Density)之间的转换公式为$\text{RMS}=\text{AWR}*\sqrt{Bandwidth}$
%	\item 跨轴灵敏度
%	\item 对齐误差
\end{itemize}