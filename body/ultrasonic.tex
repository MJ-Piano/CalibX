\section{超声波雷达}
超声波传感器主要用于停车辅助和自动停车系统。这些ADAS传感器位于前后保险杠盖内,利用反射的高频声波来识别汽车附近的人、汽车和其他物体。模拟式技术方案:这是目前市场上使用最广泛的技术路线。其主要优点是产品成本低,使其在大规模应用中具有明显的竞争优势。然而,模拟式技术方案的主要缺点是容易受到外界环境的干扰,这在某些应用场景中可能会影响雷达的性能和可靠性。四线式数位技术方案:在数位式技术路线中,四线式数位方案是目前最常用的。这种方案的主要优点是信号数字化,可以极大地提高雷达的抗干扰能力。然而,这种技术方案的成本较高,技术难度也大。由于现阶段的工艺水平限制,多数应用仍然采用四线式做法。二线式和三线式数位技术方案:这两种技术方案都属于数位式技术路线,与四线式数位方案相比,它们在某些方面可能有所优势,但目前在市场上的应用还不够广泛。超声波是声波,是机械波,要传播必须要有传播介质,不能在真空中传播。超声波传播速率跟声波一样,空气中都是340m/s左右,频率一般高于20kHz,在空气中波长一般短于2cm,因为频率超出了人耳的听觉范围,所以被称为超声。而超声波用于测距和定位的原理,就是蝙蝠在夜间飞行所需要的技能,蝙蝠以脉冲形式发射超声波,通过接收反射的回波,进行回声定位。常用探头的工作频率有 40kHz, 48kHz 和 58kHz 三种。一般来说,频率越高,灵敏度越高,但水平与垂直方向的探测角度就越小,故一般采用 40kHz 的探头。超声波雷达防水、防尘,即使有少量的泥沙遮挡也不影响。探测范围在 0.1-3 米之间,而且精度较高,因此非常适合应用于泊车。
%http://www.ait-prime.com/index.php?c=show&id=120
%https://www.youhangtec.com/news/113.html
\begin{figure}[H]
	\centering
	\includegraphics[width=1.0\textwidth]{figure/ultrasonic/parking-assistant}
	\caption{超声波雷达停止辅助}
	\label{fig:parking-assistant}
\end{figure}
UPA超声波雷达,Ultrasonic Parking Assistant,频率较高, 58kHz,精度高,感测距离较短。APA超声波雷达,Automatic Parking Assistant,频率较低,40kHz,精度一般,但感测距离较长,一般超过三米。所以自动泊车的过程,就是汽车低速巡航时,使用超声波雷达感知周围环境,寻找空车位,并将汽车自动泊入车位。另外横向感测的应用,主要用于车辆行驶过程中,提醒平行车道中车辆的距离。
1、优势:超声波的能量消耗较缓慢,在介质中传播的距离比较远,穿透性强,测距的方法简单,成本低。
2、劣势:超声波雷达在速度很高情况下测量距离有一定的局限性,这是因为超声波的传输速度很容易受天气情况的影响,在不同的天气情况下,超声波的传输速度不同,而且传播速度较慢,当汽车高速行驶时,使用超声波测距无法跟上汽车的车距实时变化,误差较大。另一方面,超声波散射角大,方向性较差,在测量较远距离的目标时,其回波信号会比较弱,影响测量精度。但是,在短距离测量中,超声波测距传感器具有非常大的优势。
%%%%%%%%%%%%%%%%%%%%%%%%%%%%%%%%%%%%%%%%%%%%%%%%%%%%%%%%%%%%%%%%%%%%%%%
%https://sensor.ofweek.com/2023-08/ART-81004-8420-30605520.html
\subsection{工作原理}
载超声波雷达在近距离检测中特别有效,尤其是在停车和低速行驶时。它们通常被安装在汽车的前、后和侧面,帮助驾驶员识别障碍物,如其他车辆、行人或物体,从而避免碰撞。这种技术对于提高道路安全和辅助驾驶员在复杂环境中驾驶至关重要。超声波雷达的工作原理是通过超声波发射装置发射超声波,并在它们与障碍物碰撞后接收反射回来的超声波,从而计算物体与车辆的距离。以下是其详细的工作机制:发射超声波:传感器会发射一系列的超声波脉冲。这些超声波通常是在人耳无法听到的频率范围内,大约在20kHz到75kHz之间。超声波的传播:这些超声波脉冲会从传感器发射出去,并在空气中传播。碰撞与反射:当这些超声波遇到障碍物(如墙壁、车辆、行人等)时,它们会被反射回来。接收反射波:传感器接收这些反射回来的超声波。计算时间差:系统会计算超声波从发射到被反射并返回到传感器所需的时间。由于超声波在空气中的传播速度是已知的(大约为343米/秒,但这会受到温度和湿度的影响),所以可以通过这个时间差来计算出障碍物与传感器之间的距离。数据处理:得到的数据会被送到车辆的中央处理单元,该单元会根据这些数据做出决策,如发出警告声、显示图形界面上的警告或自动调整车辆的速度。反馈给驾驶员:系统会通过声音、仪表盘上的指示灯或其他图形界面向驾驶员提供反馈,告诉他们障碍物的位置和距离。
\begin{figure}[H]
	\centering
	\includegraphics[width=1.0\textwidth]{figure/ultrasonic/principle}
	\caption{超声波雷达工作原理}
	\label{fig:ultrasonic-principle}
\end{figure}

\subsection{技术路线}
\begin{figure}[H]
	\centering
	\includegraphics[width=1.0\textwidth]{figure/ultrasonic/roadmap}
	\caption{超声波雷达技术路线}
	\label{fig:ultrasonic-roadmap}
\end{figure}
%
%\subsubsection{3D毫米波雷达}
%
%\subsubsection{4D毫米波雷达}
%
%\subsubsection{毫米波雷达测量}
%
%\paragraph{位置测量}\mbox{}\\
%\paragraph{速度测量}\mbox{}\\
%\paragraph{方向测量}\mbox{}\\
%%%%%%%%%%%%%%%%%%%%%%%%%%%%%%%%%%%%%%%%%%%%%%%%%%%%%%%%%%%%%%%%%%%%%%%%
%
%
%\subsection{毫米波雷达与相机标定}
