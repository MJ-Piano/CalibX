\pagenumbering{arabic}
\section{数学基础}
\subsection{符号定义}
\begin{itemize}
	\item 加粗大写字母表示矩阵,如旋转矩阵$\mathbf{R}$
	\item 加粗小写字母表示向量,如速度$\mathbf{v}$
	\item 大写字母表示固定标量,如光速$C$
	\item 小写字母表示标量,如时间$t$
	\item 花体表示坐标系,如相机坐标系$\mathcal{C}$
	\item 空心字体表示集合,如实数集$\mathbb{R}$
\end{itemize}
\subsubsection{坐标系}
\begin{itemize}
	\item 世界坐标系$\mathcal{W}$
	\item IMU坐标系$\mathcal{I}$,一般也是本体坐标$\mathcal{B}$
	\item 相机坐标系$\mathcal{C}$
	\item 雷达坐标系$\mathcal{L}$
	\item 机器人坐标系$\mathcal{R}$,指机器人的重心
	\item 轮式里程计坐标系$\mathcal{O}$,指两轮轴的中心
	\item GPS坐标系$\mathcal{G}$
\end{itemize}
\subsection{位姿及其表示方式}

\subsection{刚体运动}
%https://pubs.cstam.org.cn/data/article/mie/preview/pdf/2018-1-75.pdf
三维空间中,刚体作定轴转动,如图\ref{fig:rigid-body-motion}所示。
\begin{figure}[H]
	\centering
	\includegraphics[width=1.0\textwidth]{figure/math/rigid-body-motion}
	\caption{三维空间中,刚体作定轴转动}
	\label{fig:rigid-body-motion}
\end{figure}
则刚体上一点$\mathbf{p}$的速度$\mathbf{v}$和加速度$\mathbf{a}$与刚体的角速度$\boldsymbol{\omega}$和角加速度$\boldsymbol{\alpha}$之间的关系式为,
\begin{equation}
	\begin{aligned}
		\mathbf{v} &= \boldsymbol{\omega} \times \mathbf{r} \\
		\mathbf{a} &= \dot{\mathbf{v}} = \dot{\boldsymbol{\omega}} \times \mathbf{r} + \boldsymbol{\omega} \times \mathbf{v} = \boldsymbol{\alpha} \times \mathbf{r} + \boldsymbol{\omega} \times \mathbf{v}
	\end{aligned}
\end{equation}

\subsection{B样条}
\subsubsection{B样条标准基本形式}
B样条(B-Spline)\cite{geneva2018research}由一系列控制点(也被称为“结点”)组成。由控制点定义的基函数的线性组合来表示连续函数。以平移为例,$t$时刻的平移$\mathbf{p}(t)$用样条表示为,
\begin{equation}
	\mathbf{p}(t)=\sum_{i=0}^N b_{i, k}(t) \mathbf{p}_i 
\end{equation}
其中,$N$是控制点的个数,$k$是样条的阶数,$b_{i, k}(t)$由De Boor-Cox递归公式\cite{de1978practical}定义,
\begin{equation}
	b_{i, 0}(t):= \begin{cases}1 & \text { if } \quad t_i \leq t<t_{i+1} \\ 0 & \text { otherwise }\end{cases}
\end{equation}
\begin{equation}
	b_{i, k}(t):=\frac{t-t_i}{t_{i+k}-t_i} b_{i, k-1}(t)+\frac{t_{i+k+1}-t}{t_{i+k+1}-t_{i+1}} b_{i+1, k-1}(t)
\end{equation}

\subsubsection{均匀B样条标准基本矩阵形式}
均匀B样条是B样条的一种特殊情形,即任意相邻控制点的时间之差相等,
\begin{equation}
	u(t)=\frac{t-t_i}{t_{i+1}-t_i}=\frac{t-t_i}{\Delta t}, u \in[0,1)
\end{equation}
根据\cite{qin1998general},De Boor-Cox递归形式可以转化为矩阵相乘,即
\begin{equation}
	\mathbf{p}(t)=\mathbf{b}(u(t))\left[\begin{array}{c}
		\mathbf{p}_0^{\top} \\
		\vdots \\
		\mathbf{p}_i^{\top} \\
		\vdots \\
		\mathbf{p}_n^{\top}
	\end{array}\right]=\left[\begin{array}{lllll}
		1 & \cdots & u(t)^i & \cdots & u(t)^{k-1}
	\end{array}\right] \mathbf{M}^k\left[\begin{array}{c}
		\mathbf{p}_0^{\top} \\
		\vdots \\
		\mathbf{p}_i^{\top} \\
		\vdots \\
		\mathbf{p}_n^{\top}
	\end{array}\right]
\end{equation}
其中,
\begin{equation}
	\mathbf{M}^k=\left[\begin{array}{cccc}
		m_{0,0} & m_{0,1} & \cdots & m_{0, n} \\
		m_{1,0} & m_{1,1} & \cdots & m_{1, n} \\
		\vdots & \vdots & \ddots & \vdots \\
		m_{k-1,0} & m_{k-1,1} & \cdots & m_{k-1, n}
	\end{array}\right]_{k \times n}
\end{equation}
\begin{equation}
	m_{i, j}=\frac{1}{(k-1) !} \mathbf{C}_{k-1}^{k-1-i} \sum_{s=j}^{k-1}\left[(-1)^{s-j} \mathbf{C}_k^{s-j}(k-s-1)^{k-1-i}\right]
\end{equation}
\begin{equation}
	\mathbf{C}_n^i=\frac{n !}{i !(n-i) !}
\end{equation}
标准形式的表示的B样条存在不连续的问题。因此,对于位姿而言,将轨迹表示成累计基本形式B样条更好,即
\begin{equation}
	\mathbf{p}(t)=\tilde{b}_{0, k}(t) \mathbf{p}_0 +\sum_{i=1}^n \tilde{b}_{i, k}(t) \left(\mathbf{p}_i-\mathbf{p}_{i-1}\right) 
\end{equation}
\begin{equation}
	\tilde{b}_{i, k}(t)=\sum_{s=i}^n b_{s, k}(t)
\end{equation}
\subsubsection{均匀B样条累计基本矩阵形式}
当相邻控制点的时间之差相等时,相邻控制点的时间之差相等为,
\begin{equation}
	\mathbf{p}(t)=\tilde{\mathbf{b}}(u(t))\left[\begin{array}{c}
		\mathbf{p}_0^{\top} \\
		\vdots \\
		\mathbf{p}_i^{\top} \\
		\vdots \\
		\mathbf{p}_n^{\top}
	\end{array}\right]=\left[\begin{array}{lllll}
		1 & \cdots & u(t)^i & \cdots & u(t)^{k-1}
	\end{array}\right] \tilde{\mathbf{M}}^k\left[\begin{array}{c}
		\mathbf{p}_0^{\top} \\
		\vdots \\
		\mathbf{p}_i^{\top} \\
		\vdots \\
		\mathbf{p}_n^{\top}
	\end{array}\right]
\end{equation}
其中,
\begin{equation}
	\tilde{\mathbf{M}}^k=\left[\begin{array}{cccc}
		\sum\limits_{s=0}^{n-1} m_{0, s} & \sum\limits_{s=1}^{n-1} m_{0, s} & \cdots & \sum\limits_{s=n-1}^{n-1} m_{0, s} \\
		\sum\limits_{s=0}^{n-1} m_{1, s} & \sum\limits_{s=1}^{n-1} m_{1, s} & \cdots & \sum\limits_{s=n-1}^{n-1} m_{1, s} \\
		\vdots & \vdots & \ddots & \vdots \\
		\sum\limits_{s=0}^{n-1} m_{k-1, s} & \sum\limits_{s=1}^{n-1} m_{k-1, s} & \cdots & \sum\limits_{s=n-1}^{n-1} m_{k-1, s}
	\end{array}\right]_{k \times n}
\end{equation}
%%%%%%%%%%%%%%%%%%%%%%%%%%%%%%%%%%%%%%%%%%%%%%%%%%%%%%%%%%%%%%

\subsubsection{均匀B样条表示轨迹}
假设位姿表示为,
\begin{equation}
	{ }_s^w \mathbf{T}=\left[\begin{array}{cc}
		{ }_s^w \mathbf{R} & { }^w \mathbf{p}_s \\
		\mathbf{0} & 1
	\end{array}\right], \quad{ }_s^w \mathbf{T} \in \mathbb{S E}(3), \quad{ }_s^w \mathbf{R} \in \mathbb{S O}(3)
\end{equation}
假设位姿的速度和角速度是恒定的,利用矩阵的指数变换和对数变换,均匀B样条定义为,
\begin{equation}
	{ }_s^w \mathbf{T}(t)=\exp \left(\tilde{b}_{0, k}(t) \log \left({ }_s^w \mathbf{T}_0\right)\right) \prod_{i=1}^n \exp \left(\tilde{b}_{i, k}(t) \log \left({ }_s^w \mathbf{T}_{i-1}^{-1} {_s^w}\mathbf{T}_i\right)\right)
\end{equation}
改方式样条定义期望具有以下性质,
\begin{itemize}
	\item 局部控制性,即允许系统以在线或者批量的形式运行
	\item 二阶连续性,以保证能够二次微分得到加速度
	\item 能够较好的近似最小关节力矩轨迹
	\item 对于刚体运动没有奇异值
\end{itemize}
累计基本形式的三次B样条满足上述性质,即
\begin{equation}
	{ }_s^w \mathbf{T}(t)=\exp \left(\tilde{B}_{0,4}(t) \log \left({ }_{i-1}^w \mathbf{T}\right)\right) \prod_{i=1}^3 \exp \left(\tilde{B}_{i, 4}(t) \log \left({ }_{i-1}^w \mathbf{T}^{-1}{ }_i^w \mathbf{T}\right)\right)
\end{equation}
展开为,
\begin{equation}\label{equ:cubic-bspline-pose}
	\begin{aligned}
		& { }_s^w \mathbf{T}(t)=\exp \left(\tilde{B}_{0,4}(t) \log \left({ }_{i-1}^w \mathbf{T}\right)\right) \exp \left(\tilde{B}_{1,4}(t) \log \left({ }_{i-1}^w \mathbf{T}^{-1}{ }_i^w \mathbf{T}\right)\right) \\
		& \quad \exp \left(\tilde{B}_{2,4}(t) \log \left({ }_i^w \mathbf{T}^{-1}{ }_{i+1}^w \mathbf{T}\right)\right) \exp \left(\tilde{B}_{3,4}(t) \log \left({ }_{i+1}^w \mathbf{T}^{-1}{ }_{i+2}^w \mathbf{T}\right)\right)
	\end{aligned}
\end{equation}
其累计矩阵形式为,
\begin{equation}\label{equ:bspline-basis}
	\begin{aligned}
		\tilde{\mathbf{B}}(u(t)) & =\left[\begin{array}{lllll}
			1 & \cdots & u(t)^i & \cdots & u(t)^{4-1}
		\end{array}\right] \tilde{\mathbf{M}}^4 \\
		& =\left[\begin{array}{llll}
			1 & u & u^2 & u^3
		\end{array}\right]\left[\begin{array}{llll}
			\sum\limits_{s=0}^{4-1} m_{0, s} & \sum\limits_{s=1}^{4-1} m_{0, s} & \sum\limits_{s=2}^{4-1} m_{0, s} & \sum\limits_{s=3}^{4-1} m_{0, s} \\
			\sum\limits_{s=0}^{4-1} m_{1, s} & \sum\limits_{s=1}^{4-1} m_{1, s} & \sum\limits_{s=2}^{4-1} m_{1, s} & \sum\limits_{s=3}^{4-1} m_{1, s} \\
			\sum\limits_{s=0}^{4-1} m_{2, s} & \sum\limits_{s=1}^{4-1} m_{2, s} & \sum\limits_{s=2}^{4-1} m_{2, s} & \sum\limits_{s=3}^{4-1} m_{2, s} \\
			\sum\limits_{s=0}^{4-1} m_{3, s} & \sum\limits_{s=1}^{4-1} m_{3, s} & \sum\limits_{s=2}^{4-1} m_{3, s} & \sum\limits_{s=3}^{4-1} m_{3, s}
		\end{array}\right]_{4 \times 4} \\
		& =\frac{1}{3 !}\left[\begin{array}{llll}
			1 & u & u^2 & u^3
		\end{array}\right]\left[\begin{array}{cccc}
			6 & 5 & 1 & 0 \\
			0 & 3 & 3 & 0 \\
			0 & -3 & 3 & 0 \\
			0 & 1 & -2 & 1
		\end{array}\right]_{4 \times 4} \\
		& =\frac{1}{3 !}\left[\begin{array}{lllll}
			6 & \left(5+3 u-3 u^2+u^3\right) & \left(1+3 u+3 u^2-2 u^3\right) & u^3
		\end{array}\right]_{1 \times 4}
	\end{aligned}
\end{equation}
代入公式(\ref{equ:cubic-bspline-pose})得到,
\begin{equation}\label{equ:cubic-bspline-pose-expand}
	\begin{aligned}
		{ }_s^w \mathbf{T}(u(t))&= \exp \left(\frac{1}{6}(6) \log \left({ }_{i-1}^w \mathbf{T}\right)\right) \exp \left(\frac{1}{6}\left(5+3 u-3 u^2+u^3\right) \log \left({ }_{i-1}^w \mathbf{T}^{-1}{ }_i^w \mathbf{T}\right)\right) \\
		& \exp \left(\frac{1}{6}\left(1+3 u+3 u^2-2 u^3\right) \log \left({ }_i^w \mathbf{T}^{-1}{ }_{i+1}^w \mathbf{T}\right)\right) \exp \left(\frac{1}{6} u^3 \log \left({ }_{i+1}^w \mathbf{T}^{-1}{ }_{i+2}^w \mathbf{T}\right)\right) \\
		&=\exp \left(\log \left({ }_{i-1}^w \mathbf{T}\right)\right) \exp \left(\frac{1}{6}\left(5+3 u-3 u^2+u^3\right) \log \left({ }_i^{i-1} \mathbf{T}\right)\right) \\
		& \quad \exp \left(\frac{1}{6}\left(1+3 u+3 u^2-2 u^3\right) \log \left({ }_{i+1}^i \mathbf{T}\right)\right) \exp \left(\frac{1}{6} u^3 \log \left({ }_{i+2}^{i+1} \mathbf{T}\right)\right)
	\end{aligned}
\end{equation}
\subsubsection{B样条微分}
公式(\ref{equ:cubic-bspline-pose-expand})可以简化为,
\begin{equation}
	\begin{aligned}
		{ }_s^w \mathbf{T}(u(t)) & ={ }_{i-1}^w \mathbf{T} \exp \left(B_0(u(t)){ }_i^{i-1} \boldsymbol{\Omega}\right) \exp \left(B_1(u(t)){ }_{i+1}^i \boldsymbol{\Omega}\right) \exp \left(B_2(u(t)){ }_{i+2}^{i+1} \boldsymbol{\Omega}\right) \\
		& ={ }_{i-1}^w \mathbf{T} \mathbf{A}_0 \mathbf{A}_1 \mathbf{A}_2
	\end{aligned}
\end{equation}
根据乘法链式法则并注意${ }_s^w \mathbf{T}_i$与时间$t$无关,则对时间$t$求一阶导得到,
\begin{equation}
	\frac{\partial}{\partial t}{ }_s^w \mathbf{T}(u(t))={ }_s^w \mathbf{T}_{i-1}\left(\dot{\mathbf{A}}_0 \mathbf{A}_1 \mathbf{A}_2+\mathbf{A}_0 \dot{\mathbf{A}}_1 \mathbf{A}_2+\mathbf{A}_0 \mathbf{A}_1 \dot{\mathbf{A}}_2\right)
\end{equation}
对时间$t$求二阶导得到,
\begin{equation}
	\begin{aligned}
		& \frac{\partial}{\partial t^2}{ }_s^w \mathbf{T}(u(t))={ }_s^w \mathbf{T}_{i-1}\left(\ddot{\mathbf{A}}_0 \mathbf{A}_1 \mathbf{A}_2+\mathbf{A}_0 \ddot{\mathbf{A}}_1 \mathbf{A}_2+\mathbf{A}_0 \mathbf{A}_1 \ddot{\mathbf{A}}_2\right. \\
		&\left.+2 \dot{\mathbf{A}}_0 \dot{\mathbf{A}}_1 \mathbf{A}_2+2 \mathbf{A}_0 \dot{\mathbf{A}}_1 \dot{\mathbf{A}}_2+2 \dot{\mathbf{A}}_0 \mathbf{A}_1 \dot{\mathbf{A}}_2\right)
	\end{aligned}
\end{equation}
以$\mathbf{A}_0$为例,讲述如何求得一阶导和二阶导的解析形式。指数展开为,
\begin{equation}
	\mathbf{A}_0=\mathbf{I}+B_0(u(t))_i^{i-1} \boldsymbol{\Omega}^{\wedge}+\frac{1}{2} B_0(u(t))^2\left({ }_i^{i-1} \boldsymbol{\Omega}^{\wedge}\right)^2+\frac{1}{6} B_0(u(t))^3\left({ }_i^{i-1} \boldsymbol{\Omega}^{\wedge}\right)^3+\cdots
\end{equation}
对时间$t$求导得到,
\begin{equation}
	\begin{aligned}
		\dot{\mathbf{A}}_0&=\dot{B}_0(u(t))_i^{i-1} \boldsymbol{\Omega}^{\wedge}+\dot{B}_0(u(t)) B_0(u(t))\left({ }_i^{i-1} \boldsymbol{\Omega}^{\wedge}\right)^2+\frac{1}{2} \dot{B}_0(u(t)) B_0(u(t))^2\left({ }_i^{i-1} \boldsymbol{\Omega}^{\wedge}\right)^3+\cdots \\
		&=\dot{B}_0(u(t)){ }_i^{i-1} \boldsymbol{\Omega}^{\wedge}\left(\mathbf{I}+B_0(u(t))\left({ }_i^{i-1} \boldsymbol{\Omega}^{\wedge}\right)+\frac{1}{2} \dot{B}_0(u(t)) B_0(u(t))^2\left({ }_i^{i-1} \boldsymbol{\Omega}^{\wedge}\right)^2+\cdots\right) \\
		&=\dot{B}_0(u(t))_i^{i-1} \mathbf{\Omega}^{\wedge} \mathbf{A}_0 
	\end{aligned}
\end{equation}
因此,
\begin{equation}
	\ddot{\mathbf{A}}_0=\dot{B}_0(u(t))_i^{i-1} \boldsymbol{\Omega}^{\wedge} \dot{\mathbf{A}}_0+\ddot{B}_0(u(t))_i^{i-1} \boldsymbol{\Omega}^{\wedge} \mathbf{A}_0
\end{equation}
其中,根据公式(\ref{equ:bspline-basis}),$\tilde{\mathbf{B}}(u(t))$对时间$t$求导得到,
\begin{equation}
	\begin{aligned}
		& \dot{\mathbf{B}}(u(t))=\frac{1}{3 !}\left[\begin{array}{llll}
			0 & \left(3-6 u+3 u^2\right) & \left(3+6 u-6 u^2\right) & 3 u^2
		\end{array}\right]_{1 \times 4} \times \frac{1}{\Delta t} \\
		& \ddot{\mathbf{B}}(u(t))=\frac{1}{3 !}\left[\begin{array}{llll}
			0 & (-6+6 u) & (6-12 u) & 6 u
		\end{array}\right]_{1 \times 4} \times \frac{1}{\Delta t^2}
	\end{aligned}
\end{equation}
对$\mathbf{A}_i$求一阶导、二阶导也有类似的结果。综上,对位姿的一阶导和二阶导为,
\begin{equation}
	\begin{aligned}
		{ }_s^w \dot{\mathbf{T}}(u(t))={ }_{i-1}^w \mathbf{T}\left(\dot{\mathbf{A}}_0 \mathbf{A}_1 \mathbf{A}_2+\mathbf{A}_0 \dot{\mathbf{A}}_1 \mathbf{A}_2\right. & \left.+\mathbf{A}_0 \mathbf{A}_1 \dot{\mathbf{A}}_2\right) \\
		{ }_s^w \ddot{\mathbf{T}}(u(t))={ }_{i-1}^w \mathbf{T}\left(\ddot{\mathbf{A}_0} \mathbf{A}_1 \mathbf{A}_2+\mathbf{A}_0 \ddot{\mathbf{A}_1} \mathbf{A}_2\right. & +\mathbf{A}_0 \mathbf{A}_1 \ddot{\mathbf{A}}_2 \\
		& \left.+2 \dot{\mathbf{A}}_0 \dot{\mathbf{A}}_1 \mathbf{A}_2+2 \mathbf{A}_0 \dot{\mathbf{A}}_1 \dot{\mathbf{A}}_2+2 \dot{\mathbf{A}}_0 \mathbf{A}_1 \dot{\mathbf{A}}_2\right)
	\end{aligned}
\end{equation}
其中,
\begin{equation}
	\begin{aligned}
		{ }_i^{i-1} \boldsymbol{\Omega} & =\log \left({ }_{i-1}^w \mathbf{T}^{-1} \underset{i}{w} \mathbf{T}\right) \\
		\mathbf{A}_j & =\exp \left(B_j\left(u(t){ }_{i+j}^{i-1+j} \mathbf{\Omega}\right)\right. \\
		\dot{\mathbf{A}}_j & =\dot{B}_j(u(t))_{i+j}^{i-1+j} \mathbf{\Omega}^{\wedge} \mathbf{A}_j \\
		\ddot{\mathbf{A}}_j & =\dot{B}_j(u(t))_{i+j}^{i-1+j} \mathbf{\Omega}^{\wedge} \dot{\mathbf{A}}_j+\ddot{B}_j(u(t))_{i+j}^{i-1+j} \mathbf{\Omega}^{\wedge} \mathbf{A}_j \\
		B_0(u(t)) & =\frac{1}{3 !}\left(5+3 u-3 u^2+u^3\right) \\
		B_1(u(t)) & =\frac{1}{3 !}\left(1+3 u+3 u^2-2 u^3\right) \\
		B_2(u(t)) & =\frac{1}{3 !}\left(u^3\right) \\
		\dot{B}_0(u(t)) & =\frac{1}{3 !} \frac{1}{\Delta t}\left(3-6 u+3 u^2\right) \\
		\dot{B}_1(u(t)) & =\frac{1}{3 !} \frac{1}{\Delta t}\left(3+6 u-6 u^2\right) \\
		\dot{B}_2(u(t)) & =\frac{1}{3 !} \frac{1}{\Delta t}\left(3 u^2\right) \\
		\ddot{B}_0(u(t)) & =\frac{1}{3 !} \frac{1}{\Delta t^2}(-6+6 u) \\
		\ddot{B}_1(u(t)) & =\frac{1}{3 !} \frac{1}{\Delta t^2}(6-12 u) \\
		\ddot{B}_2(u(t)) & =\frac{1}{3 !} \frac{1}{\Delta t^2}(6 u)
	\end{aligned}
\end{equation}
%
%\begin{equation}
%	\mathbf{T}(t)=\overline{\mathbf{T}}_{i-1} \prod_{j=1}^{k-1} \operatorname{Exp}\left(b_j(u(t)) \boldsymbol{\Omega}_{i+j-1}\right)
%\end{equation}
%其中,$u(t)=\left(t-t_i\right) / \Delta t \in[0,1)$,$t \in\left[t_i, t_{i+1}\right)$并决定索引$i$。
%\begin{equation}
%	b(u)=\mathbf{C}\left[\begin{array}{lllll}
%		1 & u & u^2 & \cdots & u^{k-1}
%	\end{array}\right]^{\top}
%\end{equation}
%$\mathbf{C} \in \mathbb{R}^{k \times k}$是一个常值矩阵.
%\begin{equation}
%	\boldsymbol{\Omega}_i=\operatorname{Log} \left(\overline{\mathbf{T}}_{i-1}^{-1} \overline{\mathbf{T}}_i\right)
%\end{equation}
%$\left\{\overline{\mathbf{T}}_l\right\}_{l=\{0, \ldots, M\}}, \overline{\mathbf{T}}_l \in \mathrm{SE}(3)$是样条的控制点,表示位姿。变换$\operatorname{Exp}:\mathbb{R}^6 \rightarrow \mathrm{SE}(3)$,即将6维向量转成位姿矩阵。其逆变换为$\operatorname{Log}:\mathrm{SE}(3) \rightarrow \mathbb{R}^6$,即将位姿矩阵转成6维向量。
\subsection{微积分}
\subsubsection{三角函数微积分}
余弦函数,
\begin{equation}\label{equ:integrate_cos}
	\begin{aligned}
		& \int \cos (b t+a) d t \\
		& \xlongequal{x=bt+a, t=\frac{x-a}{b}} \int \cos (x) d (\frac{x-a}{b}) \\
		& = \int \frac{\cos (x)}{b} d x \\
		& = \frac{sin(x)}{b} + C \\
		& = \frac{sin(bt+a)}{b} + C
%		& \overset{x=bt+a, t=\frac{x-a}{b}, dt=\frac{dx}{b}}{=} \int \frac{\cos (x)}{b} d x
	\end{aligned}	
\end{equation}
正弦函数,
\begin{equation}\label{equ:integrate_sin}
	\begin{aligned}
		& \int \sin (b t+a) d t \\
		& \xlongequal{x=bt+a, t=\frac{x-a}{b}} \int \sin (x) d (\frac{x-a}{b}) \\
		& = \int \frac{\sin (x)}{b} d x \\
		& = -\frac{cos(x)}{b} + C \\
		& = -\frac{cos(bt+a)}{b} + C
		%		& \overset{x=bt+a, t=\frac{x-a}{b}, dt=\frac{dx}{b}}{=} \int \frac{\cos (x)}{b} d x
	\end{aligned}	
\end{equation}
\subsection{线性问题}


\subsection{非线性问题}
%https://www.rtklib.com/prog/manual_2.4.2.pdf
假设测量$\mathbf{z}$与状态变量$\mathbf{x}$之间通过非线性函数$\mathbf{h}(\cdot)$联系,即
\begin{equation}\label{equ:nonlinear-model}
	\mathbf{z} = \mathbf{h}(\mathbf{x})+\mathbf{v}
\end{equation}
其中$\mathbf{v}$表示测量噪声,一般建模成高斯噪声。公式(\ref{equ:nonlinear-model})的一阶泰勒展开式为,
\begin{equation}\label{equ:nonlinear-taylor-model}
	\mathbf{z} \approx \mathbf{h}(\mathbf{x}_0)+\mathbf{J}(\mathbf{x}-\mathbf{x}_0) +\mathbf{v}
\end{equation}
其中,$\mathbf{x}_0$是当前状态变量的初值,$\mathbf{J}$是$\mathbf{h}(\mathbf{x})$在$\mathbf{x}=\mathbf{x}_0$处相对于$\mathbf{x}$的偏导数,即
\begin{equation}
	\mathbf{J}=\left.\frac{\partial \mathbf{h}(\mathbf{x})}{\partial \mathbf{x}}\right|_{\mathbf{x}=\mathbf{x}_0}
\end{equation}
问题(\ref{equ:nonlinear-taylor-model})转变为线性问题,即
\begin{equation}\label{equ:nonlinear2linear-model}
	\mathbf{z} - \mathbf{h}(\mathbf{x}_0) = \mathbf{J}(\mathbf{x}-\mathbf{x}_0) +\mathbf{v}
\end{equation}
类似于线性问题的求解,
\begin{equation}
	\mathbf{J}^T \mathbf{W} \mathbf{J}\left(\hat{\mathbf{x}}-\mathbf{x}_0\right)=\mathbf{J}^T \mathbf{W}\left(\mathbf{z}-\mathbf{h}\left(\mathbf{x}_0\right)\right)
\end{equation}
因此得到状态变量的估计值,
\begin{equation}
	\hat{\mathbf{x}}=\mathbf{x}_0+\left(\mathbf{J}^T \mathbf{W} \mathbf{H}\right)^{-1} \mathbf{J}^T \mathbf{W}\left(\mathbf{z}-\mathbf{h}\left(\mathbf{x}_0\right)\right)
\end{equation}
一般初值离收敛值较远的情况下,需要多次迭代以得到更好的状态估计值。完整的高斯-牛顿法求解非线性问题如算法(\ref{alg:gauss-newton})所示。\\
\SetKwRepeat{Do}{do}{while}%
%\setcounter{algorithm}{3}
\begin{algorithm}[H]
	\caption{高斯-牛顿法求解非线性问题}%算法名字
	\label{alg:gauss-newton}
	\LinesNumbered %要求显示行号
	\KwIn{状态变量初始值$\mathbf{x}_0$}%输入参数
	\KwOut{状态变量收敛值$\hat{\mathbf{x}}$}%输出
	设置状态收敛阈值$\epsilon_x$,设置迭代次数$i=0$和最大迭代次数$N$,设置代价函数$\mathbf{r}_i = 0$及代价变化量阈值$\epsilon_r$\; %\;用于换行
	\Do{$\left\| \mathbf{x}_{i+1}-\mathbf{x}_i\right\|_2>\epsilon$ and $i<N$}{
		计算测量估计$\mathbf{h}(\mathbf{x}_i)$和雅克比矩阵$\mathbf{J}_i=\left.\frac{\partial \mathbf{h}(\mathbf{x})}{\partial \mathbf{x}}\right|_{\mathbf{x}=\mathbf{x}_i}$\;
		求解状态变量${\mathbf{x}_{i+1}}=\mathbf{x}_i+\left(\mathbf{J}^T_i \mathbf{W} \mathbf{H}\right)^{-1} \mathbf{J}^T_i \mathbf{W}\left(\mathbf{z}-\mathbf{h}\left(\mathbf{x}_i\right)\right)$ \;
		计算代价$\mathbf{r}_{i+1} = \mathbf{z} - \mathbf{h}(\mathbf{x}_{i+1})$\;
		\If{$\left\| \mathbf{r}_{i+1}-\mathbf{r}_i\right\|_2<\epsilon_r$}{
			break\;
		}
		更新迭代次数$i=i+1$\;
	}
\end{algorithm}
其中,设置了三种常见的判断迭代收敛的方法,即更新前后状态变量的变化量小于阈值、迭代次数小于最大迭代次数、更新前后代价的变化量小于阈值。

\subsection{最小二乘问题}
最小二乘问题是一种特殊非线性问题,实践中遇到的很多优化问题都是最小二乘问题,其形式为,
\begin{equation}\label{equ:least-square-problem}
	\min_{\mathbf{x}}\frac{1}{2}\sum_{i}\Vert \mathbf{z}_i - \mathbf{h}_i(\mathbf{x}) \Vert^2
\end{equation}
该问题为无约束最小二乘问题。其中,$\mathbf{x}$是待优化的状态变量,$\mathbf{z}_i$是第$i$个观测,通过非线性函数$\mathbf{h}_i(\cdot)$与状态变量联系。Ceres\cite{AgarwalCeresSolver2022}可以很方便的求解上述最小二乘问题,并支持状态变量的上下界约束,即
\begin{equation}\label{equ:least-square-problem-bound}
	\begin{aligned}
		& \min_{\mathbf{x}}\frac{1}{2}\sum_{i}\Vert \mathbf{z}_i - \mathbf{h}_i(\mathbf{x}) \Vert^2 \\
		& \mbox{s.t.}\quad l_i<x_i<u_i
	\end{aligned}
\end{equation}
该问题为界约束最小二乘问题。其中,$l_i, u_i$分别是状态变量$x_i$的下界和上界。当$l_i=u_i$时,上下界约束变成常值约束,即Ceres支持设置某些状态变量为常值,即,
\begin{equation}\label{equ:least-square-problem-constant}
	\begin{aligned}
		& \min_{\mathbf{x}}\frac{1}{2}\sum_{i}\Vert \mathbf{z}_i - \mathbf{h}_i(\mathbf{x}) \Vert^2 \\
		& \mbox{s.t.}\quad x_i=C
	\end{aligned}
\end{equation}
该问题为常值约束最小二乘问题。其中,$C$是常量。特别地,对于加权最小二乘问题有,
\begin{equation}\label{equ:weight-least-square-problem}
	\min_{\mathbf{x}}\frac{1}{2}\sum_{i}\Vert \mathbf{W}_i(\mathbf{z}_i - \mathbf{h}_i(\mathbf{x})) \Vert^2
\end{equation}
其中,$\mathbf{W}_i$是观测$\mathbf{z}_i$的权重矩阵。
\subsubsection{因子图}
除了Ceres,加权最小二乘问题还可以用因子图的方式来构建和求解(当然,因子图不仅仅能求解加权最小二乘问题),如图\ref{fig:factor-graph}所示。
\begin{figure}[H]
	\centering
	\includegraphics[width=0.7\textwidth]{figure/math/factor_graph}
	\caption{因子图。黑色圆圈节点$\mathbf{x}_{i}$表示待优化的状态变量。$\mathbf{e}_{ij}$表示节点$\mathbf{x}_{i}$与$\mathbf{x}_{j}$之间存在观测约束}
	\label{fig:factor-graph}
\end{figure}
以G2O\cite{kummerle2011g}为例,其求解的问题为,
\begin{equation}\label{equ:g2o-least-square-problem}
	\begin{aligned}
		\mathbf{F}(\mathbf{x}) & =\sum_{\langle i, j\rangle \in \mathcal{C}} \underbrace{\mathbf{e}\left(\mathbf{x}_i, \mathbf{x}_j, \mathbf{z}_{i j}\right)^{\top} \boldsymbol{\Omega}_{i j} \mathbf{e}\left(\mathbf{x}_i, \mathbf{x}_j, \mathbf{z}_{i j}\right)}_{\mathbf{F}_{i j}} \\
		\mathbf{x}^* & =\underset{\mathbf{x}}{\operatorname{argmin}} \mathbf{F}(\mathbf{x}) .
	\end{aligned}
\end{equation}
其中,$\mathbf{e}\left(\mathbf{x}_i, \mathbf{x}_j, \mathbf{z}_{i j}\right)$表示节点$\mathbf{x}_{i}$与$\mathbf{x}_{j}$之间存在观测约束$\mathbf{z}_{ij}$,即
\begin{equation}
	\mathbf{e}\left(\mathbf{x}_i, \mathbf{x}_j, \mathbf{z}_{i j}\right) = \mathbf{z}_{ij} - \mathbf{h}_{ij}(\mathbf{x}_i, \mathbf{x}_j)
\end{equation}
$\boldsymbol{\Omega}_{i j}$是节点$\mathbf{x}_{i}$与$\mathbf{x}_{j}$之间观测的信息矩阵,是实对称正定矩阵,对信息矩阵进行Cholesky分解有,
\begin{equation}
	\boldsymbol{\Omega}_{i j} = \mathbf{L}_{ij}\mathbf{L}_{ij}^\top
\end{equation}
则公式(\ref{equ:g2o-least-square-problem})可以写成,
\begin{equation}\label{equ:least-square-problem}
	\min_{\mathbf{x}}\frac{1}{2}\sum_{i}\Vert \mathbf{L}_{ij} \mathbf{e}\left(\mathbf{x}_i, \mathbf{x}_j, \mathbf{z}_{i j}\right) \Vert^2
\end{equation}

\subsection{三角函数相关公式}
积化和差公式,
\begin{equation}\label{equ:product2sum}
	\begin{aligned}
		\sin \alpha \cos \beta & =\frac{1}{2}[\sin (\alpha+\beta)+\sin (\alpha-\beta)] \\
		\cos \alpha \sin \beta & =\frac{1}{2}[\sin (\alpha+\beta)-\sin (\alpha-\beta)] \\
		\cos \alpha \cos \beta & =\frac{1}{2}[\cos (\alpha+\beta)+\cos (\alpha-\beta)] \\
		\sin \alpha \sin \beta & =-\frac{1}{2}[\cos (\alpha+\beta)-\cos (\alpha-\beta)]
	\end{aligned}
\end{equation}
和差化积公式,
\begin{equation}\label{equ:sum2product}
	\begin{aligned}
		& \sin \alpha+\sin \beta=2 \sin \frac{\alpha+\beta}{2} \cos \frac{\alpha-\beta}{2} \\
		& \sin \alpha-\sin \beta=2 \cos \frac{\alpha+\beta}{2} \sin \frac{\alpha-\beta}{2}
	\end{aligned}
\end{equation}
合角公式,
\begin{equation}\label{equ:sum-angle}
	\begin{aligned}
		& \sin (\alpha+\beta)=\sin \alpha \cos \beta+\cos \alpha \sin \beta \\
		& \sin (\alpha-\beta)=\sin \alpha \cos \beta-\cos \alpha \sin \beta \\
		& \cos (\alpha+\beta)=\cos \alpha \cos \beta-\sin \alpha \sin \beta \\
		& \cos (\alpha-\beta)=\cos \alpha \cos \beta+\sin \alpha \sin \beta
	\end{aligned}
\end{equation}

\subsection{信号处理}
\subsubsection{互相关}
%https://en.wikipedia.org/wiki/Cross-correlation
对于两个连续实函数$f(t)$和$g(t)$,其互相关定义为,
\begin{equation}
	f(t) \star g(t) = \int_{-\infty}^{\infty}f(t)g(t+\tau)dt
\end{equation}
如果$f(t)$和$g(t)$均是周期为$T$的连续周期函数,则积分区间$\left[-\infty, \infty\right]$可以替换成任意的宽度为$T$的积分区间,即
\begin{equation}\label{equ:cross-product-cp}
	f(t) \star g(t) = \int_{t_0}^{t_0+T}f(t)g(t+\tau)dt
\end{equation}











