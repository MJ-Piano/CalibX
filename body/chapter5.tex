\section{标定板}
\subsection{Apriltag}

\subsection{RandomGrid}
\subsubsection{创建标定板}
\subsubsection{特征点检测}
假设标定板背景是白色,圆是黑色的。
自适应二值化基于积分图。通过比较当前点的灰度值与其周围图像块的灰度值均值来确定其二值化结果,如图\ref{fig:AdaptiveThreshold}所示。具体见算法\ref{alg:AdaptiveThreshold}。\\
\begin{algorithm}[H]
	\caption{自适应二值化}%算法名字
	\label{alg:AdaptiveThreshold}
	\LinesNumbered %要求显示行号
	\KwIn{灰度图宽w,高h,灰度图I,积分图intI,阈值$\tau$,半径rad,最小差别min\_d}%输入参数
	\KwOut{二值化图O(0或者255)}%输出
	$\tau = 0.7, rad = w/30, min\_d=20$ \;
	\For{$j=0; j<h; ++j$}{
		y1 = max(1, j-rad) \;
		y2 = min(h-1, j+rad) \;
		\For{$i=0; i<w; ++i$}{
			x1 = max(1, i-rad) \;
			x2 = min(w-1, i+rad) \;
			count = (x2-x1)*(y2-y1) \;
			intIy2 = intI + y2*w \;
			intIy1m1 = intI + (y1-1)*w \;
			sum = intIy2[x2] - intIy1m1[x2] - intIy2[x1-1] + intIy1m1[x1-1] \;
			avg = sum/count \;
			id = j*w+i \;
			out[id] = (I[id] $< \tau$*(avg-min\_d)) ? 0 : 255
		}
	}
\end{algorithm}
\begin{figure}[h]
	\centering
	\includegraphics[width=0.7\textwidth]{figure/cp5/integrate_graph}
	\caption{基于积分图自适应二值化算法结果。}
	\label{fig:AdaptiveThreshold}
\end{figure}
打标签如图\ref{fig:label},具体见算法\ref{alg:Label}。实际上就是把属于同一个圆的像素标记相同的标签。
\begin{figure}[h]
	\centering
	\includegraphics[width=0.7\textwidth]{figure/cp5/label}
	\caption{基于二值图的打标签算法结果。}
	\label{fig:label}
\end{figure}
\begin{algorithm}[h]
	\caption{打标签算法}%算法名字
	\label{alg:Label}
	\LinesNumbered %要求显示行号
	\KwIn{灰度图宽w,高h,积分图I,标签图label,标签向量labels}%输入参数
	\KwOut{标签图label,标签向量labels}%输出
	标签图label所有值都置为-1 \;
	\For{$d=0; d<max(w, h); ++d$}{
		\If{$d<w$}{
			\For{$r=0; r<min(d, h-1); ++r$}{
				AssignLabel(w, h, I, label, labels, r, d, 0);
			}
		}
		\If{$d<h$}{
			\For{$c=0; c<min(d, w-1); ++c$}{
				AssignLabel(w, h, I, label, labels, d, c, 0);
			}
		}
	}
%%%%%%%%%%%%%%%%%%%%%%%%%%%%%%%%%%%%
	\SetKwFunction{FMain}{AssignLabel}
	\SetKwProg{Fn}{Function}{:}{}
	\Fn{\FMain{$w,h,I,label,labels,r,c$}}{
		\If{I[r * w + c] == 0}{
			labelr = label + r * w ;
			labelrm1 = labelr - w \;
			lup = r $>$ 0 ? RootLabel(labels, labelrm1[c]) : -1 \;
			lleft = c $>$ 0 ? RootLabel(labels, labelr[c - 1]) : -1 \;
			\uIf{$lup >= 0 \&\& lleft >= 0 \&\& lup != lleft$}{
				labelr[c] = lup \;
				labels[lup].size += labels[lleft].size + 1\;
				labels[lleft].equiv = lup\;
				labels[lup].bbox.Insert(labels[lleft].bbox)\;
			}\uElseIf{lup $>=$ 0}{
				labelr[c] = lup;
				++labels[lup].size\;
				labels[lup].bbox.Insert(c, r)\;
			}\uElseIf{lleft $>=$ 0}{
				labelr[c] = lleft;
				++labels[lleft].size\;
				labels[lleft].bbox.Insert(c, r);
			}\Else{
				PixelClass pc = \{-1, IRectangle(c, r, c, r), 1\}\;
				labels.push\_back(pc)\;
				labelr[c] = labels.size() - 1\;
			}
		}  
	}
	\textbf{End Function} \\
%%%%%%%%%%%%%%%%%%%%%%%%%%%%%%%%%
	\SetKwFunction{FMain}{RootLabel}
	\SetKwProg{Fn}{Function}{:}{}
	\Fn{\FMain{$label,labels$}}{
		\If{$label >= 0$}{
			while (labels[label].equiv $>=$ 0)
			label = labels[label].equiv\;
		}  
		\textbf{return} $label; $ 
	}
	\textbf{End Function}
\end{algorithm}
一个标签向量labels保存了一个圆的信息,包含像素个数、圆的外接矩形等。算法通过面积、长宽比、像素密度等对圆进行过滤,即
\begin{equation}
\left\{
\begin{array}{lr}
	area_{min} \le area \le area_{max} \\
	aspect_{min} \le aspect = \frac{w}{h} \le aspect_{max} \\
	density_{min} \le density = \frac{pixels}{area}
\end{array}
\right.
\end{equation}
其中,$area=\mbox{外接矩形的长w*宽h}, \mbox{pixels是圆像素个数}$。
\begin{itemize}
	\item \textbf{椭圆拟合}
\end{itemize}
二次曲线可以表达成如下的形式\cite{ConicSections},
\begin{equation}\label{equ:conic-discriminant}
Q(x, y)=A x^{2}+B x y+C y^{2}+D x+E y+F=0
\end{equation}
写成矩阵的形式为,
\begin{equation}
\mathbf{x}^{T} \mathbf{C} \mathbf{x}=0
\end{equation}
其中,$\mathbf{x}=(x,y,1)^T$。
\begin{equation}\mathbf{C} =\left(\begin{array}{ccc}
A & B / 2 & D / 2 \\
B / 2 & C & E / 2 \\
D / 2 & E / 2 & F
\end{array}\right)\end{equation}
注意,$\mathbf{C}$使对称矩阵。根据\cite{ouellet2009precise},存在对偶的二次曲线,满足,
\begin{equation}
l^{T} \mathbf{C}^* l=0
\end{equation}
其中,$l=(a,b,c)^T$是正切于二次曲线的直线, $\mathbf{C}^*=\mathbf{C}^{-1}$。注意,由于$\mathbf{C}$使对称矩阵,所有$\mathbf{C}^*$也是对称矩阵。以椭圆为例,其正切曲线如图\ref{fig:dual_conic}所示。\\
\begin{figure}[h]
	\centering
	\includegraphics[width=0.7\textwidth]{figure/cp5/dual_conic}
	\caption{垂直于椭圆梯度向量的直线与椭圆相切。}
	\label{fig:dual_conic}
\end{figure}
记$\Theta=\left\{A^{*}, B^{*}, C^{*}, D^{*}, E^{*}, F^{*}\right\}$是$\mathbf{C}^*$的参数向量,则估计$\Theta$能通过最小化以下目标函数来实现,
\begin{equation}\label{equ:conic_least_square}
\Phi(\Theta)=\sum_{i \in R} \omega_{i}\left(l_{i}^{T} \mathbf{C}^{*}(\Theta) l_{i}\right)^{2}
\end{equation}
其中$R$是二次曲线的正切线集合,$\omega_{i}$为该正切线权重。由于直线的尺度是欠定的,固定向量模长$\|(a,b)^T\|=1$,公式\ref{equ:conic_least_square}可以化为,
\begin{equation}\label{equ:conic_least_square-1}
\left[\sum_{i \in R} \omega_{i}^{2} K_{i} K_{i}^{T}\right][\Theta]=0
\end{equation}
其中,$K_i$是有切线的系数组成,即$K_{i}=\left[a_{i}^{2}, a_{i} b_{i}, b_{i}^{2}, a_{i} c_{i}, b_{i} c_{i}, c_{i}^{2}\right]^{T}$。在约束$\|\Theta\|=1$下,公式\ref{equ:conic_least_square-1}使用SVD分解能够求解。然而,该约束不具有欧式变换不变性。具体到椭圆的情形,对二次曲线的判别式\ref{equ:conic-discriminant}可以添加$4AC-B^2=1$的约束。此外,通过对$\mathbf{C}$求逆得,
\begin{equation}\mathbf{C}^* =\left(\begin{array}{ccc}
\frac{(E^2 - 4*C*F)}{-4|\mathbf{C}|} & \frac{(2*B*F - D*E)}{-4|\mathbf{C}|} &   \frac{-(B*E - 2*C*D)}{-4|\mathbf{C}|}\\
\frac{(2*B*F - D*E)}{-4|\mathbf{C}|} & \frac{(D^2 - 4*A*F)}{-4|\mathbf{C}|} &  \frac{(2*A*E - B*D)}{-4|\mathbf{C}|}\\
\frac{ -(B*E - 2*C*D)}{-4|\mathbf{C}|}  & \frac{(2*A*E - B*D)}{-4|\mathbf{C}|} & \frac{(B^2 - 4*A*C)}{-4|\mathbf{C}|}
\end{array}\right)\end{equation}
%\begin{equation}
%\left\{
%\begin{array}{lr}
%\mathbf{C}^*_{00}=(E^2 - 4*C*F)/(-4|\mathbf{C}|) \\
%\mathbf{C}^*_{01}=(2*B*F - D*E)/(-4|\mathbf{C}|) \\
%\mathbf{C}^*_{02}=-(B*E - 2*C*D)/(-4|\mathbf{C}|)\\
%\mathbf{C}^*_{10}=(2*B*F - D*E)/(-4|\mathbf{C}|) \\
%\mathbf{C}^*_{11}=(D^2 - 4*A*F)/(-4|\mathbf{C}|) \\
%\mathbf{C}^*_{12}=(2*A*E - B*D)/(-4|\mathbf{C}|)\\
%\mathbf{C}^*_{20}=-(B*E - 2*C*D)/(-4|\mathbf{C}|)\\
%\mathbf{C}^*_{21}=(2*A*E - B*D)/(-4|\mathbf{C}|) \\
%\mathbf{C}^*_{22}=(B^2 - 4*A*C)/(-4|\mathbf{C}|)
%\end{array}
%\right.
%\end{equation}
其中,$|\mathbf{C}|$为行列式值,即
\begin{equation}
	|\mathbf{C}|=A*C*F - (C*D^2)/4 - (B^2*F)/4 - (A*E^2)/4 + (B*D*E)/4
\end{equation}
对偶系数$F^*$为,
\begin{equation}
F^{*}=\frac{1}{4|\mathbf{C}|}\left(4 A C-B^{2}\right)=\frac{1}{|\mathbf{C}|}\left|\begin{array}{cc}
A & B / 2 \\
B / 2 & C
\end{array}\right|
=\frac{1}{4|\mathbf{C}|}
\end{equation}
由于$\mathbf{C}^*$尺度不定,因此通过约束$\mathbf{C}^*$的尺度(即$|\mathbf{C}|=\frac{1}{4}$)能够使$F^{*}=1$即,
\begin{equation}
\left\{
\begin{aligned}
	&\Theta^{\prime}=\left(A^{\prime *}, B^{\prime *}, C^{\prime *}, D^{\prime *}, E^{\prime *}\right) \\
	&\left(A^{\prime *}, B^{\prime *}, C^{\prime *}, D^{\prime *}, E^{\prime *}, 1\right)
	=\lambda * \Theta
\end{aligned}
\right.
\end{equation}
$\Theta^{\prime}$满足,
\begin{equation}
\left[\sum_{i \in R} \omega_{i}^{2} K_{i}^{\prime} K_{i}^{T}\right]\left[\Theta^{\prime}\right]=\sum_{i \in R}-\omega_{i}^{2} K_{i}^{\prime} c_{i}^{2}
\end{equation}
其中,$K^{\prime}$是$K$中的前五个元素,即$K^{\prime}=\left[a_{i}^{2}, a_{i} b_{i}, b_{i}^{2}, a_{i} c_{i}, b_{i} c_{i}\right]^{T}$