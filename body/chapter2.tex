\section{相机标定}
\subsection{相机模型}

\subsubsection{普通相机}

\subsubsection{鱼眼相机}
Calib-X支持的广角/鱼眼相机模型为pinhole-equidistant。该模型源自Kannala-Brandt Camera Model\cite{kannala2006generic}(简称为\textbf{KB}模型)。
\begin{enumerate}[(1)]
	\item \textbf{KB正向模型}
\end{enumerate}

\begin{itemize}
	\item \textbf{投影模型}
\end{itemize}
假设$\theta$ 是入射光线和光轴的夹角,$r$是主点到像点的距离。满足,
\begin{equation}
r(\theta)=k_{1} \theta+k_{2} \theta^{3}+k_{3} \theta^{5}+k_{4} \theta^{7}+k_{5} \theta^{9}+\ldots
\end{equation}
假设$\Phi=(\theta, \varphi)^{\top}$是用来描述入射光线的方向的向量,则
\begin{equation}
\mathbf{x}=
\left(\begin{array}{l}
x \\
y
\end{array}\right)=r(\theta)\left(\begin{array}{c}
\cos \varphi \\
\sin \varphi
\end{array}\right)=\mathcal{F}(\Phi)
\end{equation}
如图\ref{kb model}所示。
\begin{figure}[h]
	\centering
	\includegraphics[width=1\textwidth]{figure/cp2/ds}
	\caption{\textbf{KB}模型。$P$}
	\label{kb model}
\end{figure}
\begin{itemize}
	\item \textbf{畸变模型}
\end{itemize}
径向畸变模型,
\begin{equation}\begin{array}{l}
\Delta_{\mathrm{r}}(\theta, \varphi)= \\
\left(l_{1} \theta+l_{2} \theta^{3}+l_{3} \theta^{5}\right)\left(i_{1} \cos \varphi+i_{2} \sin \varphi+i_{3} \cos 2 \varphi+i_{4} \sin 2 \varphi\right)
\end{array}\end{equation}
切向畸变模型,
\begin{equation}\begin{array}{l}
\Delta_{\mathrm{t}}(\theta, \varphi)= \\
\left(m_{1} \theta+m_{2} \theta^{3}+m_{3} \theta^{5}\right)\left(j_{1} \cos \varphi+j_{2} \sin \varphi+j_{3} \cos 2 \varphi+j_{4} \sin 2 \varphi\right)
\end{array}\end{equation}
畸变模型共包含14个变量。
\begin{itemize}
	\item \textbf{完整模型} 
\end{itemize}
\begin{equation}\label{equ:equi-distort}
\begin{aligned}
\mathbf{x}_{\mathrm{d}}&=\mathbf{x}+\mathbf{s} \\
&=
r(\theta) \mathbf{u}_{r}(\varphi)+\Delta_{r}(\theta, \varphi) \mathbf{u}_{r}(\varphi)+\Delta_{t}(\theta, \varphi) \mathbf{u}_{\varphi}(\varphi) \\
&=\mathcal{D}\left(\mathbf{x}_{\mathrm{d}}\right)
\end{aligned}
\end{equation}
其中,$\mathbf{u}_{r}(\varphi)=(\cos \varphi, \sin \varphi)^{\top}$和$\mathbf{u}_{\varphi}(\varphi)$是径向和切向方向的单位向量,
\begin{equation}\mathbf{s}=\mathcal{S}(\Phi)=\Delta_{r}(\theta, \varphi) \mathbf{u}_{r}(\varphi)+\Delta_{t}(\theta, \varphi) \mathbf{u}_{\varphi}(\varphi)\end{equation}
像素坐标为,
\begin{equation}\label{equ:equi-projection}
\mathbf{m}=
\left(\begin{array}{l}
u \\
v
\end{array}\right)=
\left[\begin{array}{cc}
m_{u} & 0 \\
0 & m_{v}
\end{array}\right]
\left(\begin{array}{l}
x_{\mathrm{d}} \\
y_{\mathrm{d}}
\end{array}\right)+\left(\begin{array}{c}
u_{0} \\
v_{0}
\end{array}\right)=\mathcal{A}\left(\mathrm{x}_{\mathrm{d}}\right)
\end{equation}
其中,$m_{u}$和$m_{v}$是水平和垂直方向单位距离包含的像素数。结合公式\ref{equ:equi-distort}和\ref{equ:equi-projection},记完整的正向相机模型为,
\begin{equation}
\mathbf{m}=\mathcal{P}_{\mathrm{c}}(\Phi)=\mathcal{A} \circ \mathcal{D} \circ \mathcal{F}
\end{equation}
%%%%%%%%%%%%%%%%%%%%%%%%%%%%%%%%%%%%%%%%%%%%%%%%%%
\begin{enumerate}[(2)]
	\item \textbf{KB反向模型}
\end{enumerate}
\begin{equation}
\Phi=\mathcal{P}_{\mathrm{c}}^{-1}(\mathbf{m})
=\mathcal{F}^{-1} \circ \mathcal{D}^{-1} \circ \mathcal{A}^{-1}
\end{equation}
其中,$\mathcal{F}^{-1}$和$\mathcal{A}^{-1}$比较容易得到解析解。难点在于计算$\mathcal{D}^{-1}$,即计算$\mathbf{s}$。$\mathbf{s}$在$\mathbf{x}_{\mathrm{d}}$附近用一阶泰勒近似,
\begin{equation}\label{equ:s}
\begin{aligned}
\mathbf{s} & \simeq\left(\mathcal{S} \circ \mathcal{F}^{-1}\right)\left(\mathbf{x}_{\mathbf{d}}\right)+\frac{\partial\left(\mathcal{S} \circ \mathcal{F}^{-1}\right)}{\partial \mathbf{x}}\left(\mathbf{x}_{\mathbf{d}}\right)\left(\mathbf{x}-\mathbf{x}_{\mathbf{d}}\right) \\
&=\mathcal{S}\left(\Phi_{\mathrm{d}}\right)-\frac{\partial \mathcal{S}}{\partial \Phi}\left(\frac{\partial \mathcal{F}}{\partial \Phi}\left(\Phi_{\mathrm{d}}\right)\right)^{-1} \mathbf{s}
\end{aligned}\end{equation}
其中,$\Phi_{\mathrm{d}}=\mathcal{F}^{-1}\left(\mathbf{x}_{\mathrm{d}}\right)$。
整理公式\ref{equ:s},
\begin{equation}
\mathcal{S}\left(\Phi_{\mathrm{d}}\right)= \left(I + \frac{\partial \mathcal{S}}{\partial \Phi}\left(\frac{\partial \mathcal{F}}{\partial \Phi}\left(\Phi_{\mathrm{d}}\right)\right)^{-1}\right) \mathbf{s}
\end{equation}
由此可得,
\begin{equation}\mathbf{s} \simeq\left(I+\frac{\partial \mathcal{S}}{\partial \Phi}\left(\Phi_{\mathrm{d}}\right)\left(\frac{\partial \mathcal{F}}{\partial \Phi}\left(\Phi_{\mathrm{d}}\right)\right)^{-1}\right)^{-1} \mathcal{S}\left(\Phi_{\mathrm{d}}\right)\end{equation}
因此,
\begin{equation}\begin{array}{l}
\mathcal{D}^{-1}\left(\mathbf{x}_{\mathrm{d}}\right) \simeq \mathbf{x}_{\mathrm{d}}- \\
\left(I+\left(\frac{\partial \mathcal{S}}{\partial \Phi} \circ \mathcal{F}^{-1}\right)\left(\mathbf{x}_{\mathrm{d}}\right)\left(\left(\frac{\partial \mathcal{F}}{\partial \Phi} \circ \mathcal{F}^{-1}\right)\left(\mathbf{x}_{\mathrm{d}}\right)\right)^{-1}\right)^{-1}\left(\mathcal{S} \circ \mathcal{F}^{-1}\right)\left(\mathbf{x}_{\mathrm{d}}\right)
\end{array}\end{equation}
\begin{enumerate}[(3)]
	\item \textbf{CalibX中使用的pinhole-equidistant模型}
\end{enumerate}
CalibX中使用的pinhole-equidistant模型其实与OpenCV中使用的fish-eye模型\cite{opencvfisheye}是一样的。
\begin{itemize}
	\item \textbf{正向模型}
\end{itemize}
假设相机坐标系下的三维点$\mathbf{p}=(x,y,z)^T$, 点$\mathbf{p}$转到归一化焦平面,
\begin{equation}
	x'=x/z, y'=y/z
\end{equation}
记三维点$\mathbf{p'}=(x',y',1)^T$。对$\mathbf{p'}$加畸变,
\begin{equation}
\left\{
\begin{array}{lr}
r=\sqrt{x'*x'+y'*y'}   \\
\theta = atan(r) \\
d(\theta) = \theta\left(1+k_1*\theta^2+k_2*\theta^4+k_3*\theta^6+k_4*\theta^8\right) \\
x_d = d(\theta)*x'/r, y_d = d(\theta)*y'/r  
\end{array}
\right.
\end{equation}
其中$\mathbf{p}_d=(x_d,y_d)^T$为畸变点。转化到图像坐标,
\begin{equation}
\left\{
\begin{array}{lr}
u=f_x*x_d+c_x \\
v=f_y*y_d+c_y 
\end{array}
\right.
\end{equation}
其中,$f_x,f_y$为焦距,$c_x,c_y$为光心。
\begin{itemize}
	\item \textbf{反向模型}
\end{itemize}
将图像坐标转到相机坐标系归一化焦平面,
\begin{equation}
\left\{
\begin{array}{lr}
x_d=(u-c_x)/f_x \\
y_d=(v-c_y)/f_y 
\end{array}
\right.
\end{equation}
对点$\mathbf{p}_d=(x_d,y_d)^T$去畸变,采用迭代的方式计算$\theta$

\subsection{单相机标定}

\subsection{多相机标定}

